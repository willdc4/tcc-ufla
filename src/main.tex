\documentclass{uflamon}          % classe base para a monografia

% ==============================================================================
% Utilizacao de pacotes
\usepackage[T1]{fontenc}         % usa fontes postscript com acentos
\usepackage[brazil]{babel}       % hifenização e títulos em português do Brasil
\usepackage[utf8]{inputenc}      % permite edição direta com acentos
\usepackage{amsmath}             % pacote da AMS para Matemática Avançada
\usepackage{amssymb}             % símbolos extras da AMS
\usepackage{latexsym}            % símbolos extras do LaTeX
\usepackage{graphicx}            % para inserção de gráficos
\usepackage{listings}            % para inserção de código
\usepackage{fancyvrb}            % para inserção de saídas de comandos
% \usepackage{enumerate}         % para personalizar lista enumeradas (incluso na classe)
\usepackage{longtable}           % para tambelas muito grandes NOVO!!!!

\usepackage{colortbl}            % cores em tabelas
\newcolumntype{Z}{|>{\columncolor[gray]{0.9}}l|} %cor cinza em células
% \usepackage{array}              % já incluso na classe
\newcolumntype{L}[1]{>{\raggedright\let\newline\\\arraybackslash\hspace{0pt}}m{#1}}
\newcolumntype{C}[1]{>{\centering\let\newline\\\arraybackslash\hspace{0pt}}m{#1}}
\newcolumntype{R}[1]{>{\raggedleft\let\newline\\\arraybackslash\hspace{0pt}}m{#1}}
\usepackage{multirow}            % para juntar duas linhas em uma só

\usepackage{multicol}            % para uso de várias colunas

% cores para os links cruzados
\usepackage{color}
\definecolor{rltred}{rgb}{0.2,0,0}
\definecolor{rltgreen}{rgb}{0,0.2,0}
\definecolor{rltblue}{rgb}{0,0,0.2}

\usepackage[
  colorlinks=true,
  urlcolor=rltblue,     % \href{...}{...} external (URL)
  filecolor=rltgreen,   % \href{...} local file
  linkcolor=rltred,     % \ref{...} and \pageref{...}
  citecolor=rltgreen,
  pdftitle={Administração de Serviços de Redes de Computadores em um Provedor de Acesso à Internet},
  pdfauthor={William dos Santos Abreu},
  pdfsubject={O objetivo deste documento é descrever as atividades realizadas como administrador de infraestrutura de Tecnologias da Informação e Comunicação (TICs).},
  pdfkeywords={1. Redes de Computadores. 2. Segurança da Informação.}
]{hyperref} % para referência cruzadas
% \usepackage{hyperref}   % para referência cruzadas
\usepackage{subfigure}    % figuras dentro de figuras
\usepackage{caption}      % remodelando o formato dos títulos de tabelas e figuras

% configuração padrão do listings   
\lstset{
  language=Java,
  extendedchars=true,
  tabsize=3,
  basicstyle=\footnotesize\ttfamily,
  stringstyle=\em,
  showstringspaces=false 
}

% para referências de acordo com a ABNT
% precisa instalar o abntex2 antes!!!
% http://abntex.codigolivre.org.br/
% comente se pretende usar outro padrão

% abnt-emphasize=bf coloca o título das bibliografias em negrito
% abnt-thesis-year=both
\usepackage[alf,abnt-etal-cite=3,abnt-etal-list=3,abnt-url-package=url,abnt-emphasize=bf]{abntex2cite}

% evite usar o hyperref com abntex, pode dar caca em urls... no linha anterior, informo
% para incluir urls usando o pacote url e não o hyperref
%
% caso queira o hyperref com abntex, comente a linha anterior e descomente a seguinte
%\usepackage[alf,abnt-etal-cite=3,abnt-etal-list=0,abnt-etal-text=emph]{abntex2cite}
%
% caso vc ainda use a versão anterior da abntex, comente a linha incluindo o abntex2cite
% e descomente a próxima linha 
%\usepackage[alf,abnt-etal-cite=3,abnt-etal-list=0,abnt-etal-text=emph]{abntcite}


% redefinindo formatação de títulos de tabelas e figuras


% ==============================================================================
% para os fãs do Word, descomente as linhas abaixo
% \sloppy %mais espaço entre as linhas
% \usepackage{identfirst}  % identando-se a primeira linha de cada seção
% \noindentfirst           % Tire o comentário para manter o padrão do LaTeX.

% ==============================================================================
% definido comandos na monografia - não é necessário na sua monografia 
% apenas para exemplificar a definição de novos comandos
\newcommand{\defs}[1]{\textsl{#1}}


% Especificando hifenizações que por ventura LaTeX não saiba fazer
% Por padrão 99,9% dos termos em português devem ser hifenizados corretamente.
\hyphenation{hardware software Li-nux am-bien-te diag-nos-ti-car coor-de-na-ção FAE-PE Recovery TelEduc Williams UFLA}

% ==============================================================================
% Dados da monografia, capa: autor, titulo, banca, etc... - SUBSTITUA DE ACORDO
% ==============================================================================
\author{William dos Santos Abreu}
\title{Administração de Serviços de Redes de Computadores em um Provedor de Acesso à Internet}
% \subtitle{}
% \engtitle{}
% \engsubtitle{}
% \edicao{}
\date{2020}
\tipo{Relatório de estágio supervisionado apresentado à Universidade Federal de Lavras, como parte das exigências do Curso de Ciência da Computação, para a obtenção do título de Bacharel.}
% use \orientador ou \orientadora quando for o caso
\orientador{Prof. DSc. Neumar Costa Malheiros}
% use \coorientador ou \coorientadora quando for o caso
% \coorientadora{} % comente se não tiver coorientador
\local{Lavras -- MG}
\bancaum{Prof. MSc. Antônio Banca Um}{UFM}
\bancadois{Prof. DSc. João Banca Dois}{FCO}
\bancatres{Profa. Esp. Eliza Banca Três}{BELMIS}
\bancaquatro{Prof. Esp. Carlos Banca Quatro}{IBGPLUS}
\defesa{30 de junho de 2020}
% ==============================================================================

% ##################################################
% Dados para Ficha catalográfica, gerada pelo sistema da Biblioteca da UFLA
% http://www.biblioteca.ufla.br/FichaCatalografica/
% dados para ficha catalográfica
% Elaboração da Ficha Catalográfica
%\preparofichacat{Ficha catalográfica elaborada pela Coordenadoria de Processos Técnicos \\ da Biblioteca Universitária da UFLA}
% primeiro autor - como na primeira linha da ficha catalográfica
%\fcautor{Abreu, William dos Santos}
% autores, separados por vírgula - na ficha catalográfica, no formato que
% vem após o título e a barra ("/")
%\fcautores{William dos Santos Abreu}
% caso trabalho seja ilustrado (figuras, gráficos, tabelas, etc.), 
% então informar por meio do comando a seguir
% caso não seja ilustrado, basta comentá-lo
%\fcilustrado{il.}
% dados da edição para a ficha 
% \fcedicao{}
% tipo do trabalho (tese, dissertação, etc.), de acordo com sistema
% de geração de ficha catalográfica
%\fctipo{Relatório de estágio (graduação)}
% ano da defesa, só precisa informar se for diferente do ano da publicação
% se forem iguais, comente a linha a seguir
% \fcdatadefesa{2020}
% preencher aqui com os dados de catalogação gerados pelo sistema
%\fccatalogacao{1. TCC. 2. Monografia. 3. Dissertação. 4. Tese. 5. Trabalho Científico – Normas. I. Universidade Federal de Lavras. II. Título.}
%\fcclasi{808.066}

% ##################################################

% \antesfichacat{\noindent Para citar este documento: \\UNIVERSIDADE FEDERAL DE LAVRAS. Biblioteca Universitária. \textbf{Manual de normalização e estrutura de trabalhos acadêmicos: TCC, monografias, dissertações e teses}. 2. ed. rev., atual. e ampl. Lavras, 2015. Disponível em: \url{http://www.biblioteca.ufla.br/wordpress/wpcontent/uploads/bdtd/manual_normalizacao_UFLA.pdf}. Acesso em: data de acesso.}

% \depoisfichacat{\noindent A reprodução e a divulgação total ou parcial deste trabalho são autorizadas, por qualquer meio convencional ou eletrônico, para fins de estudo e pesquisa, desde que citada a fonte.\\
% \newline
% {\small Este documento possui páginas em branco para facilitar a impressão frente-e-verso.}}

%##################################################

%##################################################

% para os exemplos do manual
%\newenvironment{exemplomanual}{
%\vspace{0.5cm}
%\noindent\begin{minipage}{\textwidth}
%\noindent\rule{\textwidth}{0.5pt}
%\vspace{-1cm}
%\begin{flushleft}
%}{
%\end{flushleft}
%\vspace{-0.6cm}
%\noindent\rule{\textwidth}{0.5pt}
%\vspace{0.3cm}
%\end{minipage}
%}

%\newenvironment{exemplomanuallista}{
%\vspace{0.3cm}
%\noindent\begin{minipage}{\textwidth - 0.5cm}
%\noindent\rule{\textwidth}{0.5pt}
%\vspace{-1cm}
%\begin{flushleft}
%}{
%\end{flushleft}
%\vspace{-0.6cm}
%\noindent\rule{\textwidth}{0.5pt}
%\vspace{0.3cm}
%\end{minipage}
%}

% por conta de alguns exemplos
%\usepackage{setspace}

%##################################################

% se vc já defendeu e tem o arquivo escaneado da folha de rosto, 
% descomente e altere o nome do arquivo
%\folhaAprovacaoAssinada{folharosto}

% Aqui começa o documento propriamente dito
\begin{document}

  \maketitle

  % Dedicatórias
  \dedic{Espaço reservado a dedicatória.}

  % Agradecimentos
  \thanks{Espaço reservado aos agradecimentos.}

  % Citação opcional
  \epigrafe{
    A informática e as telecomunicações serão para o século XXI\\
    o que as rodovias foram para o século XX. (Bill Clinton)
  }

  % palavras-chave
  \palchaves{Redes de Computadores.\; Segurança Computacional.\; IPv4.\; CGNAT.\; Firewall.\; VPN.}

  \resumo{O objetivo deste documento é descrever como são realizadas as atividades de administrador de infraestrutura de Tecnologias da Informação e Comunicação (TICs) no centro de operações de rede (NOC) de um provedor de acesso à internet (ISP), através de uma visão técnico-científica do assunto. O trabalho mostra como é dimensionada uma rede IPv4 e como é implementado CGNAT (recurso à escassez de endereços enquanto o IPv6 não é uma tecnologia de ponta-a-ponta). Também descreve como é feita a utilização dos recursos de segurança computacional aplicados às redes para controle de acesso, utilizando-se de firewall e de VPN para limitar o acesso a equipamentos conforme o princípio do privilégio mínimo. As atividades foram desenvolvidas como estágio supervisionado no NOC da Minasnet Telecomunicações Ltda, operadora de internet que presta serviço em 19 cidades no sul de Minas Gerais.}
  
  % Resumo deve conter de 150 a 500 palavras

  % keywords devem vir antes do abstract
  \keywords{Summary. Words. Representative.} % keywords
  \abstract{The abstract should contain representative words of the work content, located below the abstract, separated by two spaces, preceded by the keyword expression. These representative words are spelled with the first letter capitalized, separated by point.}

  % ##################################################

  % Dados do guia
  %\begin{titlepage}
  %\pagestyle{empty}
  %\renewcommand{\baselinestretch}{1}
  %\enlargethispage{1.5cm}
  %\input{reitoria}
  %\cleardoublepage
  %\end{titlepage}

  % ##################################################

  % descomente para habilitar a lista desejada
  % \listofilustracoes  % - Não vou precisar
  % \listoffigures
  % \listofgraficos
  % \listoftables
  % \listofquadros      % - Não vou precisar
  % \listofexemplos     % - Não vou precisar
  % \listofteoremas     % - Não vou precisar
  % \begin{center}
  \normalsize{\textbf{LISTA DE SIGLAS}}
\end{center}

\vspace{1mm}

\begin{center}
  \begin{tabular}{ m{3cm} m{10cm} }
    IP & Internet Protocol \\ 
    IPv4 & Internet Protocol version 4 \\ 
    ISP & Internet Service Provider \\ 
    NOC & Network Operations Center \\ 
    OLT & Optical Line Termination \\ 
    SSH & Secure Shell \\ 
    TCP & Transport Control Protocol \\ 
    TICs & Tecnologias da Informação Comunicação \\ 
    VLSM & Variable-Length Subnet Masking \\ 
    VPN & Virtual Private Network \\ 
    VoIP & Voice over IP \\ 
  \end{tabular}
\end{center}

  \tableofcontents

  \clearpage

  \pagestyle{ufla}

  % ==============================================================================
  % incluindo os capitulos
  % \chapter{INTRODUÇÃO}

    Este documento descreve as atividades executadas durante o estágio supervisionado realizado na empresa Minasnet Telecomunicações Ltda, doravante Minasnet.

\section{Contextualização do assunto}

    No curso de Bacharelado em Ciência da Computação da Universidade Federal de Lavras, o estudante tem a oportunidade de aprender os aspectos teóricos que fundamentam as vastas tecnologias digitais que são indispensáveis à sociedade contemporânea, as Tecnologias da Informação Comunicação (TICs). Dentro das TICs, destaca-se uma das maiores e mais poderosas ferramentas desenvolvida pelo ser humano - a \textbf{internet}, destaque neste trabalho.

    Apesar de na universidade existirem atividades e projetos em que o estudante possa obter conhecimento prático dos tópicos abordados em sala de aula, através de trabalhos práticos em laboratórios ou simuladores, de projetos de extensão acadêmica ou mesmo em empresas juniores, o estudante não conseguirá uma visão ampla da dimensão do assunto, por estar lidando com casos particulares e restritos. Por isso é importante a realização do estágio em uma organização, onde se pode aprender e desenvolver habilidades práticas em um ambiente em operação que presta serviço a toda a comunidade, seja em nível local, regional, nacional ou internacional.
    
    Sendo mais específico nas áreas de conhecimento da Ciência da Computação, este relatório de estágio tem seu objetivo em Redes de Computadores, principalmente abordando como é feita a operação e manutenção do serviço de internet, descrevendo o funcionamento de um provedor de internet e o trabalho dos administradores e dos analistas de infraestrutura de redes na organização.

\section{Caracterização do ambiente de trabalho}

    A Minasnet é um ISP (\textit{Internet Service Provider}, Provedor de Serviço de Internet em tradução livre) sediado na cidade de Perdões que leva internet banda larga para 19 cidades no sul de Minas Gerais, sendo uma empresa fundada no ano de 2006. As atividades de estágio foram realizados, majoritariamente, no NOC (\textit{Network Operations Center}, Centro de Operações de Rede em tradução livre) da empresa, sendo que algumas tarefas foram realizadas em campo e outras remotamente.

    O NOC da Minasnet consiste em um escritório onde trabalham os colaboradores da empresa. A equipe interna consiste em: Gerente, Projetista de Rede, Administrador de Redes, Analista de Redes, Analista de Sistemas e Estagiário. O estágio foi realizado no período de março de 2019 até junho de 2020, com uma carga horária de 30 horas de trabalho semanais, cumpridas em uma jornada flexível.  

    Para provisionar acesso à internet aos clientes, o ISP mantém toda uma infraestrutura física em operação, composta por roteadores, switches, terminadores ópticos (OLT) e rádios digitais. Estes equipamentos são configurados, gerenciados e monitorados pelos membros do NOC, sendo utilizados, conforme fornecido pelo fabricante, aplicações gráficas desktop ou via web para gerência e configuração do dispositivo ou então acesso por linha de comando através de protocolos SSH ou Telnet. O monitoramento é feito em tempo real através de plataforma de software configurada para apresentar dados em forma de gráfico e enviar notificações com alertas críticos através de bot de mensageiro instantâneo.
    
    Exemplificando, a aplicação desktop utilizadas para equipamentos Mikrotik é o Winbox, como também está disponível acesso aos equipamentos da marca através de página web, terminal Telnet ou SSH. Para monitoramento é utilizado o Zabbix, com disparo de mensagens através do Telegram, além do Video Wall no NOC para monitoramento em tempo real da rede (gráficos de consumo de banda e alertas de interface desconectada por exemplo) para detecção ou previsão de problemas.

    A comunicação oficial da empresa é feita através do mensageiro Telegram, do e-mail institucional e dos ramais VoIP (telefone IP). A gestão das tarefas realizadas pela equipe interna utiliza do Kanban, aplicado à ferramenta Trello.

\section{Organização e estrutura deste documento}

    Adiante, o documento apresenta todo o referencial teórico que fundamenta o que foi desenvolvido nas atividades do estágio, englobando os conceitos básicos de redes TCP/IP e VLSM dentro da administração de serviços de redes, bem como fundamentos de segurança computacional aplicados em redes. Em seguida, são detalhadas as atividades realizadas durante o trabalho como estagiário no NOC da operadora, especificamente a alocação de endereços IPv4 e implementação de uma camada de segurança com Firewall e VPN. Por fim, o trabalho é sumarizado na conclusão e é feita discussão de sua relação com a graduação.

  % \include{elementos}
  % \include{conclusao}

  % ==============================================================================
  % Incluindo bibliografia
  % \bibliographystyle{plain}         % estilo para labels em numeros
  % \bibliographystyle{alpha}         % estilo para labels em iniciais
  \bibliographystyle{abntex2-alf}     % estilo para referências usando ABNT, precisa instalar o abntex para usar!!!

  % inclui Referências Bibliográficas
  % inclui Referências Bibliográficas
  % \referencias
  % \bibliography{refbib}               % arquivo exemplo refbib.bib
  % ==============================================================================
  % Incluindo anexos numerados com letras maiusculas.
  % \apendices
  % \include{apendice1}

  % ==============================================================================
  % Fim do texto
\end{document}
