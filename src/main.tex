\documentclass{uflamon}          % classe base para a monografia

% ==============================================================================
% Utilizacao de pacotes
\usepackage[T1]{fontenc}         % usa fontes postscript com acentos
\usepackage[brazil]{babel}       % hifenização e títulos em português do Brasil
\usepackage[utf8]{inputenc}      % permite edição direta com acentos
\usepackage{amsmath}             % pacote da AMS para Matemática Avançada
\usepackage{amssymb}             % símbolos extras da AMS
\usepackage{latexsym}            % símbolos extras do LaTeX
\usepackage{graphicx}            % para inserção de gráficos
\usepackage{listings}            % para inserção de código
\usepackage{fancyvrb}            % para inserção de saídas de comandos
% \usepackage{enumerate}         % para personalizar lista enumeradas (incluso na classe)
\usepackage{longtable}           % para tabelas muito grandes NOVO!!!!

\usepackage{colortbl}            % cores em tabelas
\newcolumntype{Z}{|>{\columncolor[gray]{0.9}}l|} %cor cinza em células
% \usepackage{array}              % já incluso na classe
\newcolumntype{L}[1]{>{\raggedright\let\newline\\\arraybackslash\hspace{0pt}}m{#1}}
\newcolumntype{C}[1]{>{\centering\let\newline\\\arraybackslash\hspace{0pt}}m{#1}}
\newcolumntype{R}[1]{>{\raggedleft\let\newline\\\arraybackslash\hspace{0pt}}m{#1}}
\usepackage{multirow}            % para juntar duas linhas em uma só

\usepackage{multicol}            % para uso de várias colunas

% cores para os links cruzados
\usepackage{color}
\definecolor{rltred}{rgb}{0.2,0,0}
\definecolor{rltgreen}{rgb}{0,0.2,0}
\definecolor{rltblue}{rgb}{0,0,0.2}

\usepackage[
  colorlinks=true,
  urlcolor=rltblue,     % \href{...}{...} external (URL)
  filecolor=rltgreen,   % \href{...} local file
  linkcolor=rltred,     % \ref{...} and \pageref{...}
  citecolor=rltgreen,
  pdftitle={Administração de Serviços de Redes de Computadores em um Provedor de Acesso à Internet},
  pdfauthor={William dos Santos Abreu},
  pdfsubject={O objetivo deste documento é descrever as atividades realizadas como administrador de infraestrutura de Tecnologias da Informação e Comunicação (TICs).},
  pdfkeywords={1. Redes de Computadores. 2. Segurança da Informação.}
]{hyperref} % para referência cruzadas
% \usepackage{hyperref}   % para referência cruzadas
\usepackage{subfigure}    % figuras dentro de figuras
\usepackage{caption}      % remodelando o formato dos títulos de tabelas e figuras

% configuração padrão do listings   
\lstset{
  language=Java,
  extendedchars=true,
  tabsize=3,
  basicstyle=\footnotesize\ttfamily,
  stringstyle=\em,
  showstringspaces=false 
}

% para referências de acordo com a ABNT
% precisa instalar o abntex2 antes!!!
% http://abntex.codigolivre.org.br/
% comente se pretende usar outro padrão

% abnt-emphasize=bf coloca o título das bibliografias em negrito
% abnt-thesis-year=both
\usepackage[alf,abnt-etal-cite=3,abnt-etal-list=3,abnt-url-package=url,abnt-emphasize=bf]{abntex2cite}

% evite usar o hyperref com abntex, pode dar caca em urls... no linha anterior, informo
% para incluir urls usando o pacote url e não o hyperref
%
% caso queira o hyperref com abntex, comente a linha anterior e descomente a seguinte
%\usepackage[alf,abnt-etal-cite=3,abnt-etal-list=0,abnt-etal-text=emph]{abntex2cite}
%
% caso vc ainda use a versão anterior da abntex, comente a linha incluindo o abntex2cite
% e descomente a próxima linha 
%\usepackage[alf,abnt-etal-cite=3,abnt-etal-list=0,abnt-etal-text=emph]{abntcite}


% redefinindo formatação de títulos de tabelas e figuras


% ==============================================================================
% para os fãs do Word, descomente as linhas abaixo
% \sloppy %mais espaço entre as linhas
% \usepackage{identfirst}  % identando-se a primeira linha de cada seção
% \noindentfirst           % Tire o comentário para manter o padrão do LaTeX.

% ==============================================================================
% definido comandos na monografia - não é necessário na sua monografia 
% apenas para exemplificar a definição de novos comandos
\newcommand{\defs}[1]{\textsl{#1}}


% Especificando hifenizações que por ventura LaTeX não saiba fazer
% Por padrão 99,9% dos termos em português devem ser hifenizados corretamente.
\hyphenation{hardware software Li-nux am-bien-te diag-nos-ti-car coor-de-na-ção FAE-PE Recovery TelEduc Williams UFLA}

% ==============================================================================
% Dados da monografia, capa: autor, titulo, banca, etc... - SUBSTITUA DE ACORDO
% ==============================================================================
\author{William dos Santos Abreu}
\title{Administração de Serviços de Redes de Computadores em um Provedor de Acesso à Internet}
% \subtitle{}
% \engtitle{}
% \engsubtitle{}
% \edicao{}
\date{2020}
\tipo{Relatório de estágio supervisionado apresentado à Universidade Federal de Lavras, como parte das exigências do Curso de Ciência da Computação, para a obtenção do título de Bacharel.}
% use \orientador ou \orientadora quando for o caso
\orientador{Prof. DSc. Neumar Costa Malheiros}
% use \coorientador ou \coorientadora quando for o caso
% \coorientadora{} % comente se não tiver coorientador
\local{Lavras -- MG}
\bancaum{Prof. MSc. Antônio Banca Um}{UFM}
\bancadois{Prof. DSc. João Banca Dois}{FCO}
\bancatres{Profa. Esp. Eliza Banca Três}{BELMIS}
\bancaquatro{Prof. Esp. Carlos Banca Quatro}{IBGPLUS}
\defesa{30 de junho de 2020}
% ==============================================================================

% ##################################################
% Dados para Ficha catalográfica, gerada pelo sistema da Biblioteca da UFLA
% http://www.biblioteca.ufla.br/FichaCatalografica/
% dados para ficha catalográfica
% Elaboração da Ficha Catalográfica
%\preparofichacat{Ficha catalográfica elaborada pela Coordenadoria de Processos Técnicos \\ da Biblioteca Universitária da UFLA}
% primeiro autor - como na primeira linha da ficha catalográfica
%\fcautor{Abreu, William dos Santos}
% autores, separados por vírgula - na ficha catalográfica, no formato que
% vem após o título e a barra ("/")
%\fcautores{William dos Santos Abreu}
% caso trabalho seja ilustrado (figuras, gráficos, tabelas, etc.), 
% então informar por meio do comando a seguir
% caso não seja ilustrado, basta comentá-lo
%\fcilustrado{il.}
% dados da edição para a ficha 
% \fcedicao{}
% tipo do trabalho (tese, dissertação, etc.), de acordo com sistema
% de geração de ficha catalográfica
%\fctipo{Relatório de estágio (graduação)}
% ano da defesa, só precisa informar se for diferente do ano da publicação
% se forem iguais, comente a linha a seguir
% \fcdatadefesa{2020}
% preencher aqui com os dados de catalogação gerados pelo sistema
%\fccatalogacao{1. TCC. 2. Monografia. 3. Dissertação. 4. Tese. 5. Trabalho Científico – Normas. I. Universidade Federal de Lavras. II. Título.}
%\fcclasi{808.066}

% ##################################################

% \antesfichacat{\noindent Para citar este documento: \\UNIVERSIDADE FEDERAL DE LAVRAS. Biblioteca Universitária. \textbf{Manual de normalização e estrutura de trabalhos acadêmicos: TCC, monografias, dissertações e teses}. 2. ed. rev., atual. e ampl. Lavras, 2015. Disponível em: \url{http://www.biblioteca.ufla.br/wordpress/wpcontent/uploads/bdtd/manual_normalizacao_UFLA.pdf}. Acesso em: data de acesso.}

% \depoisfichacat{\noindent A reprodução e a divulgação total ou parcial deste trabalho são autorizadas, por qualquer meio convencional ou eletrônico, para fins de estudo e pesquisa, desde que citada a fonte.\\
% \newline
% {\small Este documento possui páginas em branco para facilitar a impressão frente-e-verso.}}

%##################################################

%##################################################

% para os exemplos do manual
%\newenvironment{exemplomanual}{
%\vspace{0.5cm}
%\noindent\begin{minipage}{\textwidth}
%\noindent\rule{\textwidth}{0.5pt}
%\vspace{-1cm}
%\begin{flushleft}
%}{
%\end{flushleft}
%\vspace{-0.6cm}
%\noindent\rule{\textwidth}{0.5pt}
%\vspace{0.3cm}
%\end{minipage}
%}

%\newenvironment{exemplomanuallista}{
%\vspace{0.3cm}
%\noindent\begin{minipage}{\textwidth - 0.5cm}
%\noindent\rule{\textwidth}{0.5pt}
%\vspace{-1cm}
%\begin{flushleft}
%}{
%\end{flushleft}
%\vspace{-0.6cm}
%\noindent\rule{\textwidth}{0.5pt}
%\vspace{0.3cm}
%\end{minipage}
%}

% por conta de alguns exemplos
%\usepackage{setspace}

%##################################################

% se vc já defendeu e tem o arquivo escaneado da folha de rosto, 
% descomente e altere o nome do arquivo
%\folhaAprovacaoAssinada{folharosto}

% Aqui começa o documento propriamente dito
\begin{document}

  \maketitle

  % Dedicatórias
  \dedic{Espaço reservado a dedicatória.}

  % Agradecimentos
  \thanks{Espaço reservado aos agradecimentos.}

  % Citação opcional
  \epigrafe{
    A informática e as telecomunicações serão para o século XXI\\
    o que as rodovias foram para o século XX. (Bill Clinton)
  }

  % palavras-chave
  \palchaves{Internet.\; Redes de Computadores.\; Segurança Computacional.}

  \resumo{O objetivo deste documento é descrever como são realizadas as atividades de administrador de infraestrutura de Tecnologias da Informação e Comunicação (TICs) no centro de operações de rede (NOC) de um provedor de acesso à internet (ISP), através de uma visão técnico-científica do assunto. O trabalho mostra como é dimensionada uma rede IPv4 e como é implementado CGNAT (recurso à escassez de endereços enquanto o IPv6 não é uma tecnologia de ponta-a-ponta). Também descreve como é feita a utilização dos recursos de segurança computacional aplicados às redes para controle de acesso, utilizando-se de firewall e de VPN para limitar o acesso a equipamentos conforme o princípio do privilégio mínimo. As atividades foram desenvolvidas como estágio supervisionado no NOC da Minasnet Telecomunicações Ltda, operadora de internet que presta serviço em 19 cidades no sul de Minas Gerais.}
  
  % Resumo deve conter de 150 a 500 palavras

  % keywords devem vir antes do abstract
  \keywords{Summary. Words. Representative.} % keywords
  \abstract{The abstract should contain representative words of the work content, located below the abstract, separated by two spaces, preceded by the keyword expression. These representative words are spelled with the first letter capitalized, separated by point.}

  % ##################################################

  % Dados do guia
  %\begin{titlepage}
  %\pagestyle{empty}
  %\renewcommand{\baselinestretch}{1}
  %\enlargethispage{1.5cm}
  %\input{reitoria}
  %\cleardoublepage
  %\end{titlepage}

  % ##################################################

  % descomente para habilitar a lista desejada
  % \listofilustracoes  % - Não vou precisar
  % \listoffigures
  % \listofgraficos
  % \listoftables       % - Não vou precisar
  \listofquadros      
  % \listofexemplos     % - Não vou precisar
  % \listofteoremas     % - Não vou precisar
  \begin{center}
  \normalsize{\textbf{LISTA DE SIGLAS}}
\end{center}

\vspace{1mm}

\begin{center}
  \begin{tabular}{ m{3cm} m{10cm} }
    IP & Internet Protocol \\ 
    IPv4 & Internet Protocol version 4 \\ 
    ISP & Internet Service Provider \\ 
    NOC & Network Operations Center \\ 
    OLT & Optical Line Termination \\ 
    SSH & Secure Shell \\ 
    TCP & Transport Control Protocol \\ 
    TICs & Tecnologias da Informação Comunicação \\ 
    VLSM & Variable-Length Subnet Masking \\ 
    VPN & Virtual Private Network \\ 
    VoIP & Voice over IP \\ 
  \end{tabular}
\end{center}

  \tableofcontents

  \clearpage

  \pagestyle{ufla}

  % ==============================================================================
  % incluindo os capitulos
  \chapter{INTRODUÇÃO}

    Este documento descreve as atividades executadas durante o estágio supervisionado realizado na empresa Minasnet Telecomunicações Ltda, doravante Minasnet.

\section{Contextualização do assunto}

    No curso de Bacharelado em Ciência da Computação da Universidade Federal de Lavras, o estudante tem a oportunidade de aprender os aspectos teóricos que fundamentam as vastas tecnologias digitais que são indispensáveis à sociedade contemporânea, as Tecnologias da Informação Comunicação (TICs). Dentro das TICs, destaca-se uma das maiores e mais poderosas ferramentas desenvolvida pelo ser humano - a \textbf{internet}, destaque neste trabalho.

    Apesar de na universidade existirem atividades e projetos em que o estudante possa obter conhecimento prático dos tópicos abordados em sala de aula, através de trabalhos práticos em laboratórios ou simuladores, de projetos de extensão acadêmica ou mesmo em empresas juniores, o estudante não conseguirá uma visão ampla da dimensão do assunto, por estar lidando com casos particulares e restritos. Por isso é importante a realização do estágio em uma organização, onde se pode aprender e desenvolver habilidades práticas em um ambiente em operação que presta serviço a toda a comunidade, seja em nível local, regional, nacional ou internacional.
    
    Sendo mais específico nas áreas de conhecimento da Ciência da Computação, este relatório de estágio tem seu objetivo em Redes de Computadores, principalmente abordando como é feita a operação e manutenção do serviço de internet, descrevendo o funcionamento de um provedor de internet e o trabalho dos administradores e dos analistas de infraestrutura de redes na organização.

\section{Caracterização do ambiente de trabalho}

    A Minasnet é um ISP (\textit{Internet Service Provider}, Provedor de Serviço de Internet em tradução livre) sediado na cidade de Perdões que leva internet banda larga para 19 cidades no sul de Minas Gerais, sendo uma empresa fundada no ano de 2006. As atividades de estágio foram realizados, majoritariamente, no NOC (\textit{Network Operations Center}, Centro de Operações de Rede em tradução livre) da empresa, sendo que algumas tarefas foram realizadas em campo e outras remotamente.

    O NOC da Minasnet consiste em um escritório onde trabalham os colaboradores da empresa. A equipe interna consiste em: Gerente, Projetista de Rede, Administrador de Redes, Analista de Redes, Analista de Sistemas e Estagiário. O estágio foi realizado no período de março de 2019 até junho de 2020, com uma carga horária de 30 horas de trabalho semanais, cumpridas em uma jornada flexível.  

    Para provisionar acesso à internet aos clientes, o ISP mantém toda uma infraestrutura física em operação, composta por roteadores, switches, terminadores ópticos (OLT) e rádios digitais. Estes equipamentos são configurados, gerenciados e monitorados pelos membros do NOC, sendo utilizados, conforme fornecido pelo fabricante, aplicações gráficas desktop ou via web para gerência e configuração do dispositivo ou então acesso por linha de comando através de protocolos SSH ou Telnet. O monitoramento é feito em tempo real através de plataforma de software configurada para apresentar dados em forma de gráfico e enviar notificações com alertas críticos através de bot de mensageiro instantâneo.
    
    Exemplificando, a aplicação desktop utilizadas para equipamentos Mikrotik é o Winbox, como também está disponível acesso aos equipamentos da marca através de página web, terminal Telnet ou SSH. Para monitoramento é utilizado o Zabbix, com disparo de mensagens através do Telegram, além do Video Wall no NOC para monitoramento em tempo real da rede (gráficos de consumo de banda e alertas de interface desconectada por exemplo) para detecção ou previsão de problemas.

    A comunicação oficial da empresa é feita através do mensageiro Telegram, do e-mail institucional e dos ramais VoIP (telefone IP). A gestão das tarefas realizadas pela equipe interna utiliza do Kanban, aplicado à ferramenta Trello.

\section{Organização e estrutura deste documento}

    Adiante, o documento apresenta todo o referencial teórico que fundamenta o que foi desenvolvido nas atividades do estágio, englobando os conceitos básicos de redes TCP/IP e VLSM dentro da administração de serviços de redes, bem como fundamentos de segurança computacional aplicados em redes. Em seguida, são detalhadas as atividades realizadas durante o trabalho como estagiário no NOC da operadora, especificamente a alocação de endereços IPv4 e implementação de uma camada de segurança com Firewall e VPN. Por fim, o trabalho é sumarizado na conclusão e é feita discussão de sua relação com a graduação.

  \chapter{REFERENCIAL TEÓRICO}

\section{Administração de serviços de redes de computadores}
    
    Citando \cite{Kurose2014}.
    
\subsection{Fundamentos de Redes de Computadores}

    Citando \cite{Nakamura2007}.

    

  \chapter{DESCRIÇÃO DAS ATIVIDADES REALIZADAS}

    Descrever a plenitude das atividades executadas no estágio seria extremamente extenso, pois o estágio na área de TI demanda a execução de inúmeras tarefas corriqueiras. Por isso, aqui serão descritas as principais atividades desenvolvidas no período do estágio em administração de serviços de redes, sendo elas a fundamentação das técnicas adquiridas através de capacitações e de treinamentos na empresa, o processo de endereçamento IP e implementação de CGNAT para provisionamento de acesso à internet, o deploy de um serviço de VPN e o desenvolvimento de regras de bloqueio por firewall. 

\section{Capacitações e treinamentos}

    Conforme definição de dicionário, treinar consiste em praticar regularmente qualquer atividade e capacitar é tornar-se apto à atividade \cite{michaelis2015}. Ou seja, primeiro vem a capacitação e por conseguinte o treinamento. Portanto, a capacitação é a primeira tarefa do estagiário dentro da organização, para nivelamento de conhecimento, e o treinamento é uma constante durante todo período de estágio supervisionado, pois o estudante estará aplicando, aprendendo e desenvolvendo habilidades práticas no ambiente da organização em todo o decorrer, sob supervisão e suporte de um profissional. A capacitação e o treinamento oferecido ao estagiário é um dos principais objetivos da atividade de estágio em si.
    
    Na Minasnet, após a admissão de qualquer colaborador, seja tanto trabalhador formal quanto estagiário, é aplicada capacitações e feito treinamentos para integrar o novato nos processos da empresa, de acordo com a função que este for assumir. Para o estagiário, nas primeiras semanas de trabalho um dos integrantes do NOC fornece uma capacitação individual expositiva, apresentando conceitos básicos de redes e da topologia do backbone da empresa, como também de procedimentos de atendimento e de suporte a clientes finais, para entender quem são os clientes da empresa. São fornecidos manuais dos equipamentos que são usados e o novato é encaminhado para um treinamento para praticar os procedimentos aprendidos. O primeiro treinamento foi montar um pequeno provedor de laboratório utilizando equipamentos Mikrotik, simulando o roteamento estático e dinâmico com OSPF em algumas RB750 e enlaces de rádio com Groove, tudo realizado no primeiro dia do estágio, para se familiarizar desde então com o RouterOS.
    
    Mais adiante, foi solicitado a matrícula e exigida a apresentação de conclusão do minicurso ofertado pela Fundação Bradesco na modalidade EaD denominado ``Fundamentos de ITIL'', no qual apresentou os conceitos básicos sobre o framework ITIL para planejamento estratégico em infraestrutura de TI, com carga horária total de 16h. O último treinamento oferecido foi o minicurso presencial e com duração de 18h denominado ``Protocolo de Roteamento OSPF e Mikrotik de Iniciante a Intermediário'', restrito aos técnicos internos da Minasnet, para treinamento de roteamento dinâmico com OSPF no Mikrotik, utilizando o simulador GNS3 bem como dispositivos reais fornecidos.
    
    Demais conhecimentos adquiridos foram obtidos com capacitação individual, em que um colega da equipe ensinava determinado procedimento e depois acompanhava a atuação, como também através de cursos e de materiais digitais compartilhados pela equipe ou pela própria iniciativa de buscar conteúdo em fóruns e em plataformas vídeo.
    
\section{Endereçamento IP e implementação de CGNAT}

    O IP é a identificação que permite que os dados trafeguem desde a origem até o destino dentro da rede mundial de computadores, por isso é necessário fornecer endereços IP para todos os dispositivos que se conectarão à internet. Em um provedor, a rede (que é um conjunto de endereços IP) é dimensionada de acordo com a quantidade de clientes que serão atendidos dentro de uma determinada área de abrangência, que geralmente é atomizada por cidades.
    
    Na Minasnet, a rede é subdividida em áreas OSPF, sendo que cada área associa uma cidade. Zonas rurais e vilarejos recebem o mesmo código de área da cidade a qual pertencem. Assim, a sub-rede é dimensionada de acordo com o tamanho da cidade e com a quantidade de clientes associados àquela área, sendo geralmente um /24 o menor prefixo atribuído a uma área e um /22 o maior. Por exemplo, a franquia de Perdões tem uma rede /22, isso significa que existem 1024 endereços públicos dedicados a atenderem os clientes dessa cidade.
    
   Os IPs são associados aos CPEs\footnote{CPE é o dispositivo gateway da rede interna de um cliente.} através de um túnel PPPoE fechado com o concentrador. Como o próprio nome sugere, o concentrador centraliza todas as conexões de camada 3 dos clientes conectados em um único equipamento. Entretanto, dependendo do hardware e da configuração do software de um equipamento concentrador, pode ser necessário a utilização de mais de um equipamento para dividir a carga do servidor PPPoE. 
   Um exemplo disso na Minasnet foi a franquia de Oliveira, uma das mais recentemente atendidas pelo ISP. A princípio existia um único concentrador PPPoE, que devido à demanda de clientes entrantes, teve sua carga dividida com uma segundo concentrador instalado junto a ele. O padrão da empresa é manter no máximo 1024 clientes em um concentrador.
   
   Para documentar a rede IP do AS\footnote{AS é o conjunto de todas as sub-redes públicas de um ISP, a Minasnet é o AS262488 e tem 8.192 IPs alocados pelo Lacnic.}, é utilizado o software PHPIPAM. Nele é possível criar o aninhamento entre sub-redes e deixar descrito qual a finalidade de cada uma delas, facilitando consultas e manipulações nas redes. Tomando o exemplo de Perdões, no PHPIPAM exite uma rede com mesmo nome da cidade, na qual estão documentados todos os IPs públicos da rede /22 dimensionada para ela como também os IPs privados utilizados, que são as redes 10.0.0.0/8, 172.16.0.0/12, 192.168.0.0/16 definidas pela RFC1918 \cite{rfc1918} e o espaço compartilhado 100.64.0.0/10 pela RFC6598 \cite{rfc6598}.
   
   Dado que a franquia de Perdões possui muito mais do que 1024 clientes, um pool\footnote{Diferente de uma rede, que tem reservado o primeiro e o último endereço para o host e para o broadcast respectivamente, no pool todos os endereços são alocados, inclusive 0 e 255.} de prefixo /22 não atenderia à demanda. A solução para isso é a utilização do artifício do NAT, disponibilizando IPs privados aos clientes e fazendo associação do par IP-porta com o IP público alvo do NAT. Redes privadas não devem ser anunciadas na internet pública e a única forma de comunicarem-se com o mundo é através do NAT. 
   
   O NAT funciona porque o que estabelece comunicação entre dois hosts é a camada de transporte, isto é, uma porta definida pelos respectivos sistemas operacionais dos hosts garante a conexão fim-a-fim na internet. Assim, a função do IP é rotear os pacotes e do TCP estabelecer a conexão HTTP, por exemplo. Um cliente com IP público dedicado tem a sua disposição 65535 portas, o que lhe daria a possibilidade de estabelecer, ao máximo, 65535 conexões simultâneas.
   
   Como no NAT os clientes finais compartilham um único IP público, todas as portas desse IP serão distribuídas entre eles, sendo que para cada solicitação de conexão é inserido na tabela de tradução de endereços o par $ ( IP_{publico}, \; Porta_{publica} ) $ associado com o par $ ( IP_{privado}, \; Porta_{privada} ) $ de forma aleatória e não conflituosa caso o par já exista na tabela, dado um tempo de vida para esse binding. Essa técnica tradicional de NAT é conhecida como masquerade, por mascarar toda rede privada por trás dele através de um único IP.
   
   Embora essa técnica resolva o problema da escassez de endereços públicos, existe uma particularidade que não pode ser omitida. De acordo com o Art. 13 do Marco Civil da Internet (Lei nº 12.965/2014), ``na provisão de conexão à internet, cabe ao administrador de sistema autônomo respectivo o dever de manter os registros de conexão (...) pelo prazo de 1 (um) ano'', sendo que um registro de conexão é definido pelo ``conjunto de informações referentes à data e hora de início e término de uma conexão à internet, sua duração e o endereço IP utilizado pelo terminal'' \cite{lei12965}. A lei ainda destaca no Art. 22 que um juiz pode ordenar ao ISP o fornecimento dos registros de conexão à internet de um determinado cliente com a finalidade de obtenção de provas para processos judiciais.
   
   Dessa forma, o ISP tem a obrigação legal de manter o rastreio sobre qual IP cada um de seus clientes utilizou para navegar na internet, pois caso ele esteja utilizando a rede para cometer algum crime, a polícia conseguirá encontrá-lo. Isso parte do princípio de que na internet todos devem ser identificados pelo IP como sendo seu endereço virtual, porém o NAT masquerade quebra essa identidade por não ser determinístico na tradução do endereço. Por isso um ISP não pode simplesmente resolver a escassez por IPs utilizando uma metodologia de NAT desenvolvida para redes SOHO\footnote{SOHO são redes de escritórios domésticos e de pequenas empresas.}, deve-se utilizar de técnica específica e determinística por questões legais, chamada por CGNAT e também conhecido por NAT da operadora.
   
   Observando os requisitos definidos pelo Art. 13 da Lei nº 12.965/2014 e relacionando-os com a infraestrutura de um concentrador, devemos registrar o timestamp de início e de fim da sessão PPPoE do cliente bem como qual o IP foi disponibilizados a ele nessa sessão. Salvar essas informações é simples quando se utiliza de um servidor RADIUS\footnote{A Minasnet utiliza o FreeRADIUS \url{https://freeradius.org}.}, pois a função básica desse serviço é autorizar a conexão de assinantes, atribuindo IP e armazenando em log as informações de acesso.

   De acordo com os requisitos para implementação do CGNAT conforme RFC6888 \cite{rfc6888}, que além de estabelecer como obrigatório comportamento entre mapeamento direto entre os pools de endereços públicos e do espaço compartilhado, está em consonâcia com o Marco Civil da Internet no quesito de logging das informações do assinante conectado. A diferença é que o Marco Civil estabelece normas tipicamente orientadas para provedores que entregam somente IP público para seus clientes, por não estabelecer uma regra de registro de número de porta nas conexões. A RFC6888 é mais ampla por contemplar as necessidades de registro de conexões através de CGNAT para garantir o rastreio da identidade dos clientes na internet, definindo como parâmetros para log:
   
   \begin{enumerate}[label=\alph*)]
       \item o protocolo de transporte;
       \item o IP interno;
       \item o IP externo de origem;
       \item a porta externa de origem;
       \item o timestamp de registro.
   \end{enumerate}

   Vale ressaltar que IP externo de origem e porta externa de origem são valores do lado do provedor, sendo que IP de destino e porta de destino seriam do lado da aplicação. Não é recomendado armazenar informações de destino dos pacotes, uma vez que isso quebraria a privacidade de navegação dos assinantes por rastrear tudo que ele tem acessado na internet \cite{rfc6888}.

   O problema de se seguir à risca os cinco tópicos definidos acima é que demandaria muito espaço em disco para a manutenção do rastreio das conexões, pois para cada novo registro na tabela de tradução de endereços seria necessário armazenar no banco de dados do log uma linha correspondente com as informações supracitadas. Seria um volume tão grande de dados que até uma consulta para atender a uma solicitação judicial poderia ser demorada, lembrando que o banco de dados mantém por 1 ano todas as informações.

   A solução desse problema é utilizar da técnica de netmap, que faz mapeamento direto entre uma faixa contígua de portas correspondentes entre os IPs públicos e privados, dimensionada com tamanho padronizado. Por exemplo, se for padronizado o tamanho da faixa de portas como 2.000, as portas $ [2000, \; 3999] $ do IP 203.0.113.1 seriam destinadas ao cliente 100.64.8.0, a subsequente $ [4000, \; 5999] $ ao 100.64.8.1 e assim por diante. Com o netmap só é necessário registrar em log o IP interno do cliente e os timestamps de início e de fim da sessão PPPoE, resultando em uma redução considerável no volume de dados, pois serão somente três campos registrados ao invés dos cinco, além de inserir novas linhas na base de dados somente quado o cliente conectar e simplesmente atualizar o registro quando desconectar, pois uma forma de implementação pode deixar o campo timestamp de desconexão null até que a desconexão aconteça e efetue o armazenamento da data-hora.

   Apesar de o netmap não gerar alto volume de dados de log de conexão dos assinantes, para que seja uma técnica suficiente na implementação do CGNAT é preciso manter documentado as regras de mapeamento de IP-porta. A utilização de uma planilha é a maneira mais prática para tal tarefa. Na Minasnet é mantida uma planilha na qual as informações são registradas de acordo com colunas contendo:

   \begin{enumerate}[label=\alph*)]
       \item nome do concentrador alvo do CGNAT;
       \item nome da franquia;
       \item sub-rede interna;
       \item sub-rede externa;
       \item timestamp de início de vigência da regra.
       \item anotações.
   \end{enumerate}
   
   Os dois primeros items são apenas por questão de organização, pois existem dezenas de concentradores no ISP e todos estão documentados nessa planilha, sendo somente as informações de mapeamento de redes interna e externa e o timestamp relevantes para eficácia do registro. As anotações tem informações sobre desativação da regra de CGNAT ou modificações nas redes, não sendo criado mais campos específicos porque o objetivo é que seja alterado o mínimo possível. Até hoje, poucas vezes houveram alterações no mapeamento das sub-redes. O Quadro \ref{tab:planilha_cgnat} exemplifica o uso da planilha para controle e documentação dos mapeamentos.

   \begin{quadro}[htb]
        \begin{center}
            \caption{Exemplo de planilha para controle de CGNAT.} 
            \label{tab:planilha_cgnat}
            \vspace{0.2cm}
        \footnotesize
            \begin{tabular}{|c|c|c|c|c|c|}
            \hline
            Concentrador & Franquia & Rede interna & Rede externa & Início & Anotações \\
            \hline
            \hline
            CON-PER-01 & Perdões & 100.64.0.0/22 & 203.0.113.0/27 & 2019-04-18 17:15 & Vigente \\
            CON-PER-02 & Perdões & 100.64.4.0/22 & 203.0.113.32/27 & 2019-05-02 12:05 & Vigente \\
            CON-PER-03 & Perdões & 100.64.8.0/22 & 203.0.113.64/27 & 2019-11-13 16:35 & Vigente \\
            \hline 
            \end{tabular}
        \end{center}
        \centering{\small Fonte: do autor (2020).} 
    \end{quadro}


    A Minasnet adota por padrão a disponibilização de 2.000 portas para cada cliente atrás do CGNAT, com o os limites inferiores e superiores das faixas de portas sendo 1536 e 65535, respectivamente. Isso significa que, para cada IP público do CGNAT do ISP, existem 32 clientes internos navegando através dele. Esse dimensionamento de 1 IP externo para 32 internos denomina a razão de compartilhamento 1:32. As portas de uso reservado (0-1023) não são usadas e a numeração começa a partir de 1536 por questões de arredondamento de cálculos.
    
    Os cálculos a seguir demonstram o dimensionamento do mapeamento de portas descrito. O primeiro passo é verificar a quantidade de portas $ \Delta $ que estão sendo dedicadas ao CGNAT por um único IP, simplesmente subtraindo os limites de portas definidos e somando 1, pois a primeira porta também é contabilizada:

    \begin{equation}
        \Delta = porta_{maior} - porta_{menor} + 1
               = 65535 - 1536 + 1
               = 64000
    \end{equation}
    
    O que resulta em 64000 portas. Como cada IP interno tem 2000 portas mapeadas para ele, então a quantidade de IPs internos $ n $ para cada IP público será:

    \begin{equation}
        n = \frac{64000}{2000}
          = 32
    \end{equation}

    O que resulta na razão 1:32.

    Como dito anteriormente, um concentrador da Minasnet normalmente é dimensionado para atender até 1024 clientes. Então, quantos IPs internos e externos seriam necessários para implementar CGNAT conforme as regras definidas até aqui? A quantidade de IPs internos é direta: 1024, pois cada IP interno atende um único cliente. A quantidade de IPs externos é obtida tirando a razão 1024 por 32, pois a razão de compartilhamento dada é 1:32, o que resulta em 32 IPs. Então, nesse concentrador deve ser criado uma mapeamento de uma rede interna com 1024 endereços (/22) para uma rede externa de 32 endereços (/27).
    
    Calcular mapeamentos para outros prefixos de rede é simples, pois seguem a mesma lógica usada no mapeamento /22 entre /27. O Quadro \ref{tab:netmap} mostra alguns dos mapeamentos que são possíveis de serem feitos seguindo a metodologia usada aqui. A demonstração pode ser feita alterando os valores correspondentes dos cálculos supracitados, em sempre será obtida a razão constante de 1:32 entre IP externo e interno.

    \begin{quadro}[htb]
        \begin{center}
            \caption{Mapeamento direto entre sub-redes internas/externas usando CGNAT 1:32.} 
            \label{tab:netmap}
            \vspace{0.2cm}
            \footnotesize
            \begin{tabular}{|c|c|c|}
            \hline
            Prefixo privado & Prefixo público & Quantidade de clientes \\
            \hline
            \hline
            /22 & /27 & 1024 \\
            /23 & /28 & 512 \\
            /24 & /29 & 256 \\
            /25 & /30 & 128 \\
            /26 & /31 & 64 \\
            /27 & /32 & 32 \\
            \hline 
            \end{tabular}
        \end{center}
        \centering{\small Fonte: do autor (2020)} 
    \end{quadro}

    \subsection{Implementação com Python e RouterOS}

    Até aqui, foi fundamentada a metodologia utilizada para o CGNAT. Agora será apresentada a implementação das regras em roteadores Mikrotik através do sistema operacional embarcado RouterOS\footnote{RouterOS \url{https://mikrotik.com/software} é o sistema operacional nativo dos equipamentos Mikrotik, sendo um software proprietário baseado em Linux.}, com geração das regras através de ferramenta desenvolvida em Python\footnote{Python \url{https://www.python.org/} é uma linguagem de programação interpretada e multiplataforma.}.
    
    O primeiro passo para o desenvolvimento prático do CGNAT é entender o que o RouterOS nos proporciona para atingir esse objetivo. Por ser baseado em Linux, a ideia básica é construir as regras de NAT através do módulo de firewall do sistema, pois o NAT é interpretado como uma regra de firewall por fazer modificação no cabeçalho do dos pacotes de dados ao alterar os valores de endereço e de porta originais.

    O módulo utilizado é a tabela NAT do firewall do RouterOS, sendo necessário configurar os seguintes parâmetros na geração das regras conforme definidos na documentação \cite{natmikrotik}:

    \begin{enumerate}[label=\alph*)]
        \item \label{nat:action} {\tt action}, sendo a ação executada pela regra o {\tt netmap}, que consiste no mapeamento direto entre as redes interna e externa;
        
        \item \label{nat:chain} {\tt chain}, sendo configurado como {\tt srcnat}, pois os pacotes alvos da regra originam-se na rede interna ao firewall; 
        
        \item \label{nat:protocol} {\tt protocol}, sendo configurados tanto TCP e UDP individualmente; 
        
        \item \label{nat:srcaddress} {\tt src-address}, sendo a sub-rede interna;
        
        \item \label{nat:toaddress} {\tt to-addresses}, sendo a sub-rede externa;
        
        \item \label{nat:toports} {\tt to-ports}, sendo a faixa de portas do IP público alocada.
    \end{enumerate}

    A única ressalva é que, para o pleno funcionamento do NAT para pacotes de protocolos de camada 3, serão criadas regras em que não estarão definidos os items (\ref{nat:protocol} e (\ref{nat:toports}.

    Tendo os parâmetros definidos, o template do comando a ser executado na CLI é dado conforme o exemplo da Figura \ref{fig:netmap_cli}.

    \begin{figure}[!htb]
        \centering
        \caption{Template do comando de configuração de netmap no RouterOS.} 
        \label{fig:netmap_cli}
        
        \begin{Verbatim}[fontsize=\small]
            /ip firewall nat 
                add action=netmap           \
                chain=srcnat                \
                protocol=tcp                \
                src-address=100.64.0.0/27   \
                to-addresses=203.0.113.0/27 \
                to-ports=1536-3535          \
                disabled=yes
        \end{Verbatim} 

        {\small Fonte: do autor (2020).} 
    \end{figure}

    O netmap é uma técnica de CGNAT horizontal.

    O exemplo utilizado no comando acima usa das sub-redes indicadas no exemplo do CON-PER-01 no Quadro \ref{tab:planilha_cgnat}. Pode-se notar que foi utilizado um prefixo /27 ao invés de um /22 no comando, isso acontece porque para que a regra de netmap faça o mapeamento direto, as redes devem ter o mesmo prefixo para que a correspondência entre os IPs seja possível. Assim, uma rede /22 deve ser subdividida em 32 sub-redes /27 para que seja feito o mapeamento com uma rede pública de prefixo /27. Para fazer o casamento entre outros valores de prefixos, o princípio é o mesmo.
    
    Acontece que, fazer geração manual de todas essas regras é um trabalho extenso e cansativo, que pode ficar sujeito a falhas humanas. Para isso, foi desenvolvido um utilitário de CLI para geração das regras para construir netmap entre qualquer prefixo, desde que respeite a razão 1:32. O software foi desenvolvido em Pyhton, o que permite que seja usado em qualquer sistema operacional que tenha Python instalado.

    O fundamento utilizado na construção do programa baseia-se na geração de uma árvore binária oriunda das subdivisões recursivas de uma rede privada de determinado prefixo, situada na raíz, até que chegue em uma camada em que as folhas tenham o mesmo prefixo que a rede pública, procedimento ilustrado na Figura \ref{fig:arvore_binaria}. A partir daí, o algoritmo obtém uma lista com todas as sub-redes formadas pelas folhas dessa árvore, e gera as regras de netmap conforme definido no template da Figura \ref{fig:netmap_cli}, incrementando o número de portas em um valor fixo de 2000. O código-fonte está disponível no APÊNDICE???

    \begin{figure}[!htb]
        \centering
        \caption{Árvore binária das subdivisões recursivas de um /22 até um /25.}
        \label{fig:arvore_binaria}
        
        \small
        \begin{tikzpicture}[->,>=stealth',level/.style={sibling distance = 7cm/#1,
  level distance = 1.5cm}] 
            \node {/22}
            child {
                node {/23}
                child {
                    node {/24}
                    child {
                        node {/25}
                        node {/25}
                    }
                    child {
                        node {/25}
                        node {/25}
                    }
                }
                child {
                    node {/24}
                    child {
                        node {/25}
                        node {/25}
                    }
                    child {
                        node {/25}
                        node {/25}
                    }
                }
            }
            child {
                node {/23}
                child {
                    node {/24}
                    child {
                        node {/25}
                        node {/25}
                    }
                    child {
                        node {/25}
                        node {/25}
                    }
                }
                child {
                    node {/24}
                    child {
                        node {/25}
                        node {/25}
                    }
                    child {
                        node {/25}
                        node {/25}
                    }
                }
            };
        \end{tikzpicture}

        {\small Fonte: do autor (2020).} 
    \end{figure}
    
    DOCUMENTAÇÃO DO CGNAT.PY
  % \include{conclusao}

  % ==============================================================================
  % Incluindo bibliografia
  % \bibliographystyle{plain}         % estilo para labels em numeros
  % \bibliographystyle{alpha}         % estilo para labels em iniciais
  \bibliographystyle{abntex2-alf}     % estilo para referências usando ABNT, precisa instalar o abntex para usar!!!

  % inclui Referências Bibliográficas
  % inclui Referências Bibliográficas
  % \referencias
   \bibliography{referencias}               % arquivo exemplo refbib.bib
  % ==============================================================================
  % Incluindo anexos numerados com letras maiusculas.
  % \apendices
  % \include{apendice1}

  % ==============================================================================
  % Fim do texto
\end{document}
