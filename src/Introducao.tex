\chapter{INTRODUÇÃO}

    Este documento descreve as atividades executadas durante o estágio supervisionado realizado na empresa Minasnet Telecomunicações LTDA, doravante Minasnet.

\section{Contextualização do assunto}

    A internet tornou-se um dos principais meios de comunicação utilizado pela população. Esse fato é sustentado pelos dados recentemente divulgados pela DataReportal, que mostram os números de usuários na internet brasileira crescendo a cada ano. No relatório de janeiro de 2020, o portal contabilizou 150,4 milhões de navegantes na rede no país, uma valor que cresceu 6\% em um ano, tendo 8.5 milhões de usuários novos em relação a 2019 \cite{datereportal2020}. O dado atual mostra uma penetração de 71\% da internet na população do Brasil, que significa que 71\% dos brasileiros têm acesso ao meio de telecomunicação.
    
    Dessa forma, manter operacional toda a infraestrutura que sustenta essa crescente demanda por acesso à internet necessita de profissionais habilitados. Por isso, existem os profissionais que administram os serviços de rede das operadoras, construindo e mantendo a internet de seus assinantes com a maior disponibilidade possível. Para cada clique na internet feito por um usuário final, existe toda uma infraestrutura que permite que os dados trafeguem através de enlaces de fibra óptica, para manter o mundo inteiro conectado, em uma fração de segundos.

\section{Caracterização do ambiente de trabalho}

    A Minasnet é um ISP (\textit{Internet Service Provider} -- Provedor de Serviço de Internet em tradução livre), sediado na cidade de Perdões, que leva internet banda larga para 21 cidades no sul de Minas Gerais até então, sendo uma empresa fundada no ano de 2006 na mesma cidade. As atividades de estágio foram realizados, majoritariamente, no NOC (\textit{Network Operations Center}, Centro de Operações de Rede em tradução livre) da empresa, sendo que algumas tarefas foram realizadas em campo e outras remotamente.

    O NOC da Minasnet conta com um escritório onde trabalham os técnicos internos da empresa. A equipe interna consiste colaboradores que desempenham os seguintes cargos: Gerente, Projetista de Rede, Administrador de Redes, Analista de Redes, Analista de Sistemas e Estagiário. O estágio foi realizado no período de março de 2019 até junho de 2020, com uma carga horária de 30 horas de trabalho semanais, cumpridas em uma jornada flexível.

    Para provisionar aos clientes acesso à internet, o ISP mantém toda uma infraestrutura física em operação, composta por roteadores, switches, terminadores ópticos (OLTs) e rádios digitais. Estes equipamentos são configurados, gerenciados e monitorados pela equipe do NOC, sendo utilizadas, conforme fornecido pelo fabricante, aplicações gráficas desktop ou via web para gerência e configuração dos dispositivos ou então acesso por terminal de comando através de protocolos SSH ou Telnet. O monitoramento é feito em tempo real através de plataforma de software configurada para apresentar dados na forma de um painel de controle (\textit{dashboard}) e enviar notificações com alertas críticos através de \textit{bot} para mensageiro instantâneo.
    
    Exemplificando, a aplicação desktop utilizada para equipamentos Mikrotik é o Winbox, como também está disponível acesso aos equipamentos da marca através de aplicação web, de terminal Telnet ou SSH. Para monitoramento é utilizado o Zabbix, com disparo de mensagens através do Telegram, além do Video Wall no NOC com um \textit{dashboard} completo para o monitoramento em tempo real da rede, a fim de detectar ou prever problemas, utilizando-se de gráficos de consumo de banda e de alertas de enlace desconectado, por exemplo.

    A comunicação oficial da empresa é feita através do mensageiro Telegram, do e-mail institucional e dos ramais VoIP (telefone IP) que cada colaborador possui. A gestão das tarefas realizadas pela equipe interna é feita através do Kanban, aplicado por meio da ferramenta Trello.

\section{Capacitações e treinamentos}

    Na Minasnet, após a admissão de qualquer colaborador, seja tanto trabalhador formal quanto estagiário, é aplicada capacitações e feito treinamentos para integrar o novato nos processos da empresa, de acordo com a função que este for assumir. Para o estagiário, nas primeiras semanas de trabalho um dos integrantes do NOC fornece uma capacitação individual expositiva, apresentando conceitos básicos de redes e da topologia do backbone da empresa, como também de procedimentos de atendimento e de suporte a clientes finais, para entender quem são os clientes da empresa. São fornecidos manuais dos equipamentos que são usados e o novato é encaminhado para um treinamento para praticar os procedimentos aprendidos. O primeiro treinamento foi montar um pequeno provedor de laboratório utilizando equipamentos MikroTik, simulando o roteamento estático e dinâmico com OSPF em algumas RB750\footnote{Roteador MikroTik RB750 \url{https://mikrotik.com/product/RB750r2}.} e enlaces de rádio com Groove\footnote{Rádio outdoor MikroTik Groove \url{https://mikrotik.com/product/RBGroove52HPnr2}.}, tudo realizado no primeiro dia do estágio, para se familiarizar desde então com o os principais equipamentos utilizados pelo provedor. 
    
    O último treinamento oferecido no período do estágio foi o minicurso presencial e com duração de 18h denominado ``Protocolo de Roteamento OSPF e Mikrotik de Iniciante a Intermediário'', restrito aos técnicos internos da Minasnet, para treinamento de roteamento dinâmico com OSPF no MikroTik, utilizando o simulador GNS3\footnote{GNS3 \url{https://www.gns3.com} é um simulador completo de redes.} bem como dispositivos reais fornecidos.

\section{Objetivos do estágio}
    
    Com todas as tarefas atribuídas ao estagiário somadas com os treinamentos oferecidos, o objetivo do estágio é capacitar o estudante para atuar profissionalmente na área de Redes de Computadores, podendo assumir o cargo de analista de infraestrutura de TI, de redes ou de segurança com mais convicção. As atividades de estágio possibilitaram ao estudante a operação e a manutenção do serviço de internet, trabalhando diretamente com o funcionamento de um provedor de internet e com o trabalho dos administradores e dos analistas de redes na organização.
    
    A princípio, o plano de trabalho consistia em realizar monitoramento dos ativos de rede e dos blocos de endereços IP listados em \textit{blacklist}, realizar análise dos procedimentos realizados pelos técnicos de atendimento e prestar suporte aos técnicos de instalação e de infraestrutura, criando relatórios, procedimentos e documentando os processos técnicos realizados. Porém, com a competência e o desempenho do estagiário na execução das atividades propostas, além do entrosamento com a equipe e com a aptidão em lidar com novas tecnologias, outras tarefas foram sendo atribuídas ao estagiário como uma forma de evolução nos procedimentos realizados.
    
    Assim, as atividades foram desempenhadas com o intuito de apoiar nas tarefas de monitoramento de incidentes na rede, contribuir na implementação de soluções para problemas identificados na operação dos serviços de rede, como também propor melhorias em processos técnicos no setor de TI da empresa e desenvolver métodos de segurança para combate a spam e outros \textit{worms} que se propagam pela internet. Além da execução de tarefas, o estágio também serviu como fonte de pesquisa de novos equipamentos de rede e de novas tecnologias que puderam ser aplicadas nos ambientes em produção.

\section{Estrutura do documento}

    Nos capítulos seguintes, o documento apresenta no Capítulo 2 o referencial teórico que fundamenta o que foi desenvolvido nas principais atividades do estágio, englobando os conceitos básicos de redes TCP/IP e VLSM dentro da administração de serviços de redes, bem como fundamentos de segurança computacional aplicados em redes. Em seguida, são detalhadas as atividades realizadas durante o trabalho como estagiário no NOC da operadora, especificamente sendo abordado no Capítulo 3 o endereçamento IPv4 e a implementação de CGNAT e por conseguinte, no Capítulo 4, é descrita a implementação de uma camada de segurança com Firewall e VPN. Por fim, no Capítulo 5 o trabalho é sumarizado e é apresentada também uma discussão de sua relação com o curso de graduação.
    
