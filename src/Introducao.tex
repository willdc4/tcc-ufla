\chapter{INTRODUÇÃO}

    Este documento descreve as atividades executadas durante o estágio supervisionado realizado na empresa Minasnet Telecomunicações Ltda, doravante Minasnet.

\section{Contextualização do assunto}

    No curso de Bacharelado em Ciência da Computação da Universidade Federal de Lavras, o estudante tem a oportunidade de aprender os aspectos teóricos que fundamentam as vastas tecnologias digitais que são indispensáveis à sociedade contemporânea, as Tecnologias da Informação e Comunicação (TICs). Dentro das TICs, destaca-se uma das maiores e mais poderosas ferramentas desenvolvida pelo ser humano: a \textbf{internet}, que é destaque neste trabalho.

    Apesar de na universidade existirem atividades e projetos em que o estudante possa obter conhecimento prático dos tópicos abordados em sala de aula, através de trabalhos práticos em laboratórios ou simuladores, de projetos de extensão acadêmica ou mesmo em empresas juniores, o estudante não conseguirá uma visão ampla da dimensão do assunto, por estar lidando com casos particulares e restritos. Por isso é importante a realização do estágio em uma organização, onde se pode aprender e desenvolver habilidades práticas em um ambiente em produção e que presta serviço a toda a comunidade, seja em nível local, regional, nacional ou internacional.
    
    Sendo mais específico nas áreas de conhecimento da Ciência da Computação, este relatório de estágio tem seu objetivo em Redes de Computadores, principalmente abordando como é feita a operação e a manutenção do serviço de internet, descrevendo o funcionamento de um provedor de internet e o trabalho dos administradores e dos analistas de infraestrutura de redes na organização.

\section{Caracterização do ambiente de trabalho}

    A Minasnet é um ISP (\textit{Internet Service Provider} -- Provedor de Serviço de Internet em tradução livre), sediado na cidade de Perdões, que leva internet banda larga para 19 cidades no sul de Minas Gerais até então, sendo uma empresa fundada no ano de 2006 na mesma cidade. As atividades de estágio foram realizados, majoritariamente, no NOC (\textit{Network Operations Center}, Centro de Operações de Rede em tradução livre) da empresa, sendo que algumas tarefas foram realizadas em campo e outras remotamente.

    O NOC da Minasnet conta com um escritório onde trabalham os técnicos internos da empresa. A equipe interna consiste colaboradores que desempenham os seguintes cargos: Gerente, Projetista de Rede, Administrador de Redes, Analista de Redes, Analista de Sistemas e Estagiário. O estágio foi realizado no período de março de 2019 até junho de 2020, com uma carga horária de 30 horas de trabalho semanais, cumpridas em uma jornada flexível.

    Para provisionar aos clientes acesso à internet, o ISP mantém toda uma infraestrutura física em operação, composta por roteadores, switches, terminadores ópticos (OLTs) e rádios digitais. Estes equipamentos são configurados, gerenciados e monitorados pela equipe do NOC, sendo utilizadas, conforme fornecido pelo fabricante, aplicações gráficas desktop ou via web para gerência e configuração dos dispositivos ou então acesso por terminal de comando através de protocolos SSH ou Telnet. O monitoramento é feito em tempo real através de plataforma de software configurada para apresentar dados na forma de um painel de controle (dashboard) e enviar notificações com alertas críticos através de bot para mensageiro instantâneo.
    
    Exemplificando, a aplicação desktop utilizada para equipamentos Mikrotik é o Winbox, como também está disponível acesso aos equipamentos da marca através de aplicação web, de terminal Telnet ou SSH. Para monitoramento é utilizado o Zabbix, com disparo de mensagens através do Telegram, além do Video Wall no NOC com um dashboard completo para o monitoramento em tempo real da rede, a fim de detectar ou prever problemas, utilizando-se de gráficos de consumo de banda e de alertas de enlace desconectado, por exemplo.

    A comunicação oficial da empresa é feita através do mensageiro Telegram, do e-mail institucional e dos ramais VoIP (telefone IP) que cada colaborador possui. A gestão das tarefas realizadas pela equipe interna é feita através do Kanban, aplicado por meio da ferramenta Trello.

\section{Capacitações e treinamentos}

    Aa.

\section{Objetivos do estágio}

    A princípio, o plano de trabalho consistia em realizar monitoramento dos ativos de rede e dos blocos de endereços IP listados em blacklist, realizar análise dos procedimentos realizados pelos técnicos de atendimento e prestar suporte aos técnicos de instalação e de infraestrutura, criando relatórios, procedimentos e documentando os processos técnicos realizados. Porém, com o entrosamento com a equipe, outras tarefas foram sendo atribuídas ao estagiário como uma forma de evolução dos procedimentos realizados.
    
    A atividade de monitoramento é constante e utiliza-se dos recursos que o Zabbix e o Grafana proporcionam, sendo uma tarefa tipicamente reativa às adversidades que a rede está sujeita, como saturação de link, rompimento de fibra, superaquecimento de dispositivos e etc. Quando constatada uma falha na rede, sempre é feita a notificação do ocorrido para a equipe de atendimento informar aos clientes, além de repassar para os administradores de rede poderem solucionar o problema.
    
    O problema da blacklist de IPs foi uma tarefa que apresentou evolução, pois por se tratar de um monitoramento de segurança da rede, acabou agregando ao estagiário competências de implementação de camadas de segurança, utilizando-se de firewall e de VPN, a fim de reduzir as consequências, uma vez que para um IP estar listado em uma blacklist algum dispositivo atrás dele está infectado com algum verme que propaga spam na internet.
    
    As tarefas de suporte técnico são realizas através de telefone, passando orientações e solucionando dúvidas da equipe externa, como também acessando equipamentos de clientes de maneira remota. Também foi uma atividade que acabou evoluindo para prestação de suporte interno de informática para empresa, a fim de garantir desempenho e segurança nos equipamentos, sendo feito levantamento de requisitos de hardware e de software para alocação de computadores para os funcionários de diferentes setores da empresa.
    
    A realização de documentação de procedimentos consiste em criar manuais para que outros colegas também consigam realizar os mesmos procedimentos, além dos relatórios feitos em planilhas para facilitar a organização dos dados e a obtenção de métricas. A evolução nesta atividade foi o uso do PHPIPAM para a documentação da rede IP, que consequentemente evoluiu para organização e automatização da configuração de CGNAT com Python.
    
    Enfim, com todas as tarefas atribuídas ao estagiário somadas com os treinamentos oferecidos, o objetivo do estágio é capacitar o estudante para atuar profissionalmente na área de Redes de Computadores, podendo assumir o cargo de analista de infraestrutura de TI, de redes ou de segurança com mais convicção.

\section{Estrutura do documento}

    Adiante, o documento apresenta um referencial teórico que fundamenta o que foi desenvolvido nas principais atividades do estágio, englobando os conceitos básicos de redes TCP/IP e VLSM dentro da administração de serviços de redes, bem como fundamentos de segurança computacional aplicados em redes. Em seguida, são detalhadas as atividades realizadas durante o trabalho como estagiário no NOC da operadora, especificamente a alocação de endereços IPv4 e implementação de uma camada de segurança com Firewall e VPN. Por fim, o trabalho é sumarizado na conclusão e é feita discussão de sua relação com a graduação.
