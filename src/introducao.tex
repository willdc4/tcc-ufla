\chapter{INTRODUÇÃO}

    Este documento descreve as atividades executadas durante o estágio supervisionado realizado na empresa Minasnet Telecomunicações Ltda, doravante Minasnet.

\section{Contextualização do assunto}

    No curso de Bacharelado em Ciência da Computação da Universidade Federal de Lavras, o estudante tem a oportunidade de aprender os aspectos teóricos que fundamentam as vastas tecnologias digitais que são indispensáveis à sociedade contemporânea, as Tecnologias da Informação Comunicação (TICs). Dentro das TICs, destaca-se uma das maiores e mais poderosas ferramentas desenvolvida pelo ser humano - a \textbf{internet}, destaque neste trabalho.

    Apesar de na universidade existirem atividades e projetos em que o estudante possa obter conhecimento prático dos tópicos abordados em sala de aula, através de trabalhos práticos em laboratórios ou simuladores, de projetos de extensão acadêmica ou mesmo em empresas juniores, o estudante não conseguirá uma visão ampla da dimensão do assunto, por estar lidando com casos particulares e restritos. Por isso é importante a realização do estágio em uma organização, onde se pode aprender e desenvolver habilidades práticas em um ambiente em operação que presta serviço a toda a comunidade, seja em nível local, regional, nacional ou internacional.
    
    Sendo mais específico nas áreas de conhecimento da Ciência da Computação, este relatório de estágio tem seu objetivo em Redes de Computadores, principalmente abordando como é feita a operação e manutenção do serviço de internet, descrevendo o funcionamento de um provedor de internet e o trabalho dos administradores e dos analistas de infraestrutura de redes na organização.

\section{Caracterização do ambiente de trabalho}

    A Minasnet é um ISP (\textit{Internet Service Provider}, Provedor de Serviço de Internet em tradução livre) sediado na cidade de Perdões que leva internet banda larga para 19 cidades no sul de Minas Gerais, sendo uma empresa fundada no ano de 2006. As atividades de estágio foram realizados, majoritariamente, no NOC (\textit{Network Operations Center}, Centro de Operações de Rede em tradução livre) da empresa, sendo que algumas tarefas foram realizadas em campo e outras remotamente.

    O NOC da Minasnet consiste em um escritório onde trabalham os colaboradores da empresa. A equipe interna consiste em: Gerente, Projetista de Rede, Administrador de Redes, Analista de Redes, Analista de Sistemas e Estagiário. O estágio foi realizado no período de março de 2019 até junho de 2020, com uma carga horária de 30 horas de trabalho semanais, cumpridas em uma jornada flexível.  

    Para provisionar acesso à internet aos clientes, o ISP mantém toda uma infraestrutura física em operação, composta por roteadores, switches, terminadores ópticos (OLT) e rádios digitais. Estes equipamentos são configurados, gerenciados e monitorados pelos membros do NOC, sendo utilizados, conforme fornecido pelo fabricante, aplicações gráficas desktop ou via web para gerência e configuração do dispositivo ou então acesso por linha de comando através de protocolos SSH ou Telnet. O monitoramento é feito em tempo real através de plataforma de software configurada para apresentar dados em forma de gráfico e enviar notificações com alertas críticos através de bot de mensageiro instantâneo.
    
    Exemplificando, a aplicação desktop utilizadas para equipamentos Mikrotik é o Winbox, como também está disponível acesso aos equipamentos da marca através de página web, terminal Telnet ou SSH. Para monitoramento é utilizado o Zabbix, com disparo de mensagens através do Telegram, além do Video Wall no NOC para monitoramento em tempo real da rede (gráficos de consumo de banda e alertas de interface desconectada por exemplo) para detecção ou previsão de problemas.

    A comunicação oficial da empresa é feita através do mensageiro Telegram, do e-mail institucional e dos ramais VoIP (telefone IP). A gestão das tarefas realizadas pela equipe interna utiliza do Kanban, aplicado à ferramenta Trello.

\section{Organização e estrutura deste documento}

    Adiante, o documento apresenta todo o referencial teórico que fundamenta o que foi desenvolvido nas atividades do estágio, englobando os conceitos básicos de redes TCP/IP e VLSM dentro da administração de serviços de redes, bem como fundamentos de segurança computacional aplicados em redes. Em seguida, são detalhadas as atividades realizadas durante o trabalho como estagiário no NOC da operadora, especificamente a alocação de endereços IPv4 e implementação de uma camada de segurança com Firewall e VPN. Por fim, o trabalho é sumarizado na conclusão e é feita discussão de sua relação com a graduação.
