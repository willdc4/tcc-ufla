\chapter{CONCLUSÃO}
\label{cap:conclusao}

    No curso de Bacharelado em Ciência da Computação da Universidade Federal de Lavras, o estudante tem a oportunidade de aprender os aspectos teóricos que fundamentam as vastas tecnologias digitais que são indispensáveis à sociedade contemporânea, as Tecnologias da Informação e Comunicação (TICs). Dentro das TICs, destaca-se uma das maiores e mais poderosas ferramentas desenvolvida pelo ser humano: a internet, que foi o destaque neste trabalho.

    O trabalho serviu como ponte entre o conhecimento acadêmico e o conhecimento prático. Cálculos de sub-redes podiam até parecer não ter sentido em listas de exercícios em disciplina da graduação, mas com os cálculos de endereçamento e de implantação de CGNAT mostraram sua importância para garantir que a camada lógica da internet funcione e permita que os assinantes naveguem pela rede. Não somente isso, adotar práticas e políticas de segurança com adoção de firewall e VPN também possibilitou a aplicação dos princípios que antes foram abordados somente em laboratório e em estudos de caso fictícios.
    
    O legado gerencial deixado pelo estágio está no monitoramento contínuo dos serviços como estratégia de inteligência de negócios na empresa, pois é através dele que é possível reagir às falhas o quanto antes, bem como prever problemas, a fim de garantir a qualidade do serviços prestados aos clientes. O trabalho abordou a parte mais técnica de operação de redes, mas o dia a dia gira em torno da gestão e do monitoramento de incidentes.
    
    Os objetivos foram alcançados sem muitas dificuldades operacionais para a execução dos mesmos, pois o estagiário já tinha uma bagagem teórica consolidada e o trabalho serviu como meio para praticar o conhecimento e gerar valor. As dificuldades foram recorrentes devido à insegurança em aplicar alterações em sistemas em produção com milhares de clientes acessando, algo normal quando se pensa que um erro pode 
   ``parar tudo'' em questão de segundos.

    Apesar de na universidade existirem atividades e projetos em que o estudante pode obter conhecimento prático dos tópicos abordados em sala de aula, através de trabalhos práticos em laboratórios ou simuladores, de projetos de extensão acadêmica ou mesmo em empresas juniores, o estudante não conseguirá uma visão ampla da dimensão do assunto, por estar lidando com casos particulares e restritos. Por isso, é importante a realização do estágio em uma organização, onde se pode aprender e desenvolver habilidades práticas em um ambiente em produção e que presta serviço a toda a comunidade, seja em nível local, regional, nacional ou internacional.
    
    Por isso a realização deste trabalho de estágio supervisionado na organização Minasnet foi importante para a maturação do estudante como profissional da área de redes, pois foi o momento quando se colocou em prática os protocolos aprendidos na disciplina de Redes de Computadores como também as arquiteturas de sistemas distribuídos, em nível de infraestrutura, vistos na disciplina de Arquitetura de Computadores e em Sistemas Distribuídos. De fato, o estágio abordou assuntos que foram além dessas disciplinas básicas da área de infraestrutura, o que demonstra uma falta de disciplinas ofertadas nessa área por parte do Departamento de Ciência da Computação. Durante o curso, foram mínimas (se não inexistentes) disciplinas eletivas sobre assuntos de redes, o que está deixando o curso de Ciência da Computação focado quase que exclusivamente na camada de aplicação do Modelo OSI, com foco em desenvolvimento de aplicações. Não se pode deixar de lado a infraestrutura dos sistemas de computação e de informação, pois é ela que sustenta todas as aplicações que estão em operação, por exemplo, na web.
    