\chapter{ENDEREÇAMENTO IP}
\label{cap:enderecamento}

    O endereço IP é a identificação que possibilita o tráfego de dados desde sua origem até o destino através da rede mundial de computadores, por isso é necessário fornecer endereços IP para todos os dispositivos que se conectam à internet. Em um provedor, a rede (que utiliza um conjunto de IPs com mesmo prefixo) é dimensionada de acordo com a quantidade de clientes que serão atendidos dentro de uma determinada área de abrangência, que geralmente é atomizada por cidades.

\section{Rede IP da Minasnet}
    
    Na Minasnet, a rede é subdividida em áreas OSPF, sendo que cada área é associada a uma cidade. Zonas rurais e vilarejos recebem o mesmo código de área da cidade a qual pertencem. Assim, a sub-rede é dimensionada de acordo com o tamanho da cidade e com a quantidade de clientes associados àquela área. Geralmente, um bloco /24 de endereços é o menor espaço de endereçamento atribuído a uma área e um bloco /22 é o maior. Por exemplo, a franquia de Perdões tem uma rede /22, isso significa que existem 1024 endereços públicos dedicados para atendimento aos clientes dessa cidade.
    
   Os IPs são associados aos CPEs (dispositivo \textit{gateway} da rede interna de um cliente) através de um túnel PPPoE fechado com o concentrador (B-RAS). Como o próprio nome sugere, o concentrador centraliza todas as conexões de camada 3 dos clientes conectados em um único equipamento. Entretanto, dependendo do hardware e da configuração do software de um equipamento concentrador, pode ser necessário a utilização de mais de um equipamento para dividir a carga do servidor PPPoE. 
   
   Na Minasnet, um exemplo disso foi a franquia de Oliveira, uma das mais recentemente atendidas pelo ISP. A princípio existia um único concentrador PPPoE que, devido à demanda de clientes entrantes, teve sua carga dividida com um segundo concentrador instalado junto a ele. O padrão atual da empresa é manter no máximo 1024 clientes em um único concentrador.
   
   Para documentar a rede IP do AS (conjunto de todas as sub-redes públicas de um ISP), é utilizado o software PHPIPAM\footnote{PHPIPAM \url{https://phpipam.net} é um software \textit{open-source} de gerenciamento de endereços de rede (IP, VLAN e etc.).}. Neste software é possível criar o aninhamento entre sub-redes e deixar descrito qual a finalidade de cada uma delas, facilitando consultas e manipulações de VLSM nessas redes dentro do próprio sistema. Tomando o exemplo de Perdões, no PHPIPAM existe uma rede com mesmo nome da cidade, na qual estão documentados todos os IPs públicos da rede /22 dimensionada para a mesma, além dos IPs privados utilizados na franquia, que estão contidos nas redes 10.0.0.0/8, 172.16.0.0/12 e 192.168.0.0/16, definidas pela RFC 1918 \cite{rfc1918}, e o espaço compartilhado 100.64.0.0/10 definido pela RFC 6598 \cite{rfc6598}. Dentro da rede IP pública, é reservada uma sub-rede para alocação de IP fixo e o restante dos IPs são alocados dinamicamente para os clientes.
   
   Dado que a franquia de Perdões possui muito mais do que 1024 clientes, um \textit{pool}\footnote{Diferente de uma rede, que tem reservado o primeiro e o último endereço para o \textit{host} e para o \textit{broadcast} respectivamente, no \textit{pool} todos os endereços são alocados, inclusive 0 e 255.} de prefixo /22 não atenderia à demanda. A solução para a escassez de endereços na franquia é a utilização do artifício do NAT, disponibilizando IPs privados aos clientes e fazendo associação do par IP-porta com o IP público de destino do NAT. Redes privadas não devem ser anunciadas na internet pública e a única forma de comunicarem-se com o mundo é através do NAT. 
   
\section{Metodologia para implementação de CGNAT}

   Observando os requisitos definidos pelo Art. 13 da Lei nº 12.965/2014, descritos na Seção \ref{sec:marco_civil}, e relacionando-os com a infraestrutura de um concentrador, devemos registrar o \textit{timestamp} de início e de fim da sessão PPPoE do cliente bem como qual o IP foi disponibilizado a ele durante essa sessão. A sessão PPPoE inicia-se no instante em que o cliente estabelece conexão e se encerra quando acontece a desconexão. Salvar essas informações é simples quando se utiliza um servidor RADIUS\footnote{A Minasnet utiliza o FreeRADIUS \url{https://freeradius.org}.}, pois a função básica desse serviço é autorizar a conexão de assinantes, atribuindo IP e armazenando o \textit{log} das informações de acesso. 

   A RFC 6888 estabelece requisitos para implementação do CGNAT de maneira segura. Um ponto importante dessa RFC é a definição obrigatória do mapeamento direto entre os \textit{pools} de endereços públicos e privado, imprescindível para que o CGNAT seja determinístico. Outro ponto importante é a manutenção de registro das informações do assinante conectado, assim como especificado pelo Marco Civil da Internet.
   
   Embora a RFC 6888 esteja em consonância com o Marco Civil da Internet, existe divergência. A diferença é que o Marco Civil estabelece normas tipicamente orientadas para provedores que entregam somente IP público para seus clientes, por não estabelecer uma regra de registro de número de porta nas conexões. A RFC 6888 é mais ampla por contemplar as necessidades de registro de conexões através de CGNAT para garantir o rastreio da identidade dos clientes na internet, definindo como parâmetros para \textit{log} \cite{rfc6888}:
   
   \begin{enumerate}[label=\alph*)]
       \item o protocolo de transporte;
       \item o IP interno;
       \item o IP externo de origem;
       \item a porta externa de origem;
       \item o \textit{timestamp} de registro.
   \end{enumerate}

   Vale ressaltar que IP externo de origem e porta externa de origem são valores do lado do provedor, sendo que IP de destino e porta de destino seriam do lado da aplicação. Não é recomendado armazenar informações de destino dos pacotes, uma vez que isso quebraria a privacidade de navegação dos assinantes por rastrear tudo que eles tem acessado na internet \cite{rfc6888}.

   O problema de se seguir à risca os cinco parâmetros definidos acima é que demandaria muito espaço em disco para a manutenção do rastreio das conexões, pois seria necessário armazenar no banco de dados de \textit{log} cada novo registro na tabela de tradução de endereços. Seria um volume tão grande de dados que até uma consulta para atender a uma solicitação judicial poderia ser demorada, lembrando que o banco de dados deve manter por 1 ano todas as informações.

   A solução desse problema é utilizar técnicas que fazem mapeamento direto entre uma faixa contígua de portas, de tamanho padronizado, para os IPs públicos e privados. Com o mapeamento de portas, só é necessário registrar em \textit{log} o IP interno do cliente e os \textit{timestamps} de início e de fim da sessão PPPoE, resultando em uma redução considerável no volume de dados, pois serão somente três campos registrados ao invés dos cinco. 
 
   Outra melhoria para redução do volume de dados é utilizar dois campos de \textit{timestamp} ao invés de um só -- um para o momento de conexão e outro para o de desconexão. Assim, a inserção de novas linhas na base de dados será necessária somente quado o cliente conectar, inserindo o \textit{timestamp} no campo de conexão e deixando o de desconexão \textit{null} enquanto o cliente estiver conectado. No momento da desconexão, basta atualizar o campo \textit{null} com a data e hora da desconexão.
 
   Apesar de o mapeamento de portas não gerar alto volume de dados de \textit{log} de conexão dos assinantes, para que seja uma técnica suficiente na implementação do CGNAT é preciso manter documentadas as regras de mapeamento de IP-porta. A utilização de uma planilha é a maneira mais prática para realizar tal tarefa. Na Minasnet, é mantida uma planilha na qual as informações são registradas de acordo com colunas contendo:

   \begin{enumerate}[label=\alph*)]
       \item nome do concentrador alvo do CGNAT;
       \item nome da franquia;
       \item sub-rede interna;
       \item sub-rede externa;
       \item \textit{timestamp} de início de vigência da regra.
       \item anotações.
   \end{enumerate}
   
   Os dois primeiros itens são apenas por questão de organização, pois existem dezenas de concentradores no ISP e todos estão documentados nessa planilha, sendo somente as informações de mapeamento de redes interna e externa e o \textit{timestamp} relevantes para eficácia do registro. As anotações tem informações sobre desativação da regra de CGNAT ou modificações nas redes, não sendo criado mais campos específicos porque o objetivo é que seja alterado o mínimo possível. Até hoje, poucas vezes houve alterações no mapeamento das sub-redes. O Quadro \ref{tab:planilha_cgnat} exemplifica o uso da planilha para controle e documentação dos mapeamentos.

   \begin{quadro}[htb]
        \begin{center}
            \caption{Exemplo de planilha para controle de CGNAT.} 
            \label{tab:planilha_cgnat}
            \vspace{0.2cm}
        \footnotesize
            \begin{tabular}{|c|c|c|c|c|c|}
            \hline
            Concentrador & Franquia & Rede interna & Rede externa & Início & Anotações \\
            \hline
            \hline
            CON-PER-01 & Perdões & 100.64.0.0/22 & 203.0.113.0/27 & 2019-04-18 17:15 & Vigente \\
            CON-PER-02 & Perdões & 100.64.4.0/22 & 203.0.113.32/27 & 2019-05-02 12:05 & Vigente \\
            CON-PER-03 & Perdões & 100.64.8.0/22 & 203.0.113.64/27 & 2019-11-13 16:35 & Vigente \\
            \hline 
            \end{tabular}
        \end{center}
        \centering{\small Fonte: do autor (2020).} 
    \end{quadro}


    A Minasnet adota por padrão a disponibilização de 2.000 portas para cada cliente atrás do CGNAT, definindo o intervalo da faixa entre 1536 e 65535. Isso significa que, para cada IP público do CGNAT do ISP, existem 32 clientes internos conectados à internet através dele. Esse dimensionamento de 1 IP externo para 32 internos denomina a razão de compartilhamento 1:32. As portas de uso reservado (0-1023) não são usadas e a numeração começa a partir de 1536 por questões de arredondamento de cálculos.
    
    Os cálculos a seguir demonstram o dimensionamento do mapeamento de portas descrito. O primeiro passo é verificar a quantidade de portas $ \Delta $ que estão sendo dedicadas ao CGNAT por um único IP, simplesmente subtraindo os limites de portas definidos e somando 1, pois a primeira porta também é contabilizada:

    \begin{equation}
        \Delta = porta_{maior} - porta_{menor} + 1
               = 65535 - 1536 + 1
               = 64000
    \end{equation}
    
    O que resulta em 64.000 portas. Como cada IP interno tem 2.000 portas mapeadas para ele, então a quantidade de IPs internos $ n $ para cada IP público será:

    \begin{equation}
        n = \frac{64000}{2000}
          = 32
    \end{equation}

    Como dito anteriormente, um concentrador da Minasnet normalmente é dimensionado para atender até 1024 clientes. De acordo com a proporção 1:32, são necessários 1024 IPs internos e 32 IPs externos  para implementação do CGNAT nesse caso. A quantidade de IPs externos é obtida tirando a razão 1024 por 32, pois a razão de compartilhamento dada é 1:32, o que resulta em 32 IPs. Então, nesse concentrador deve ser criado uma mapeamento de uma rede interna com 1024 endereços (/22) para uma rede externa de 32 endereços (/27).
    
    Calcular mapeamentos para outros prefixos de rede é simples, pois seguem a mesma lógica usada no mapeamento /22 entre /27. O Quadro \ref{tab:netmap} mostra alguns dos mapeamentos que são possíveis de serem feitos seguindo a metodologia usada aqui. A demonstração pode ser feita alterando os valores correspondentes dos cálculos supracitados, em sempre será obtida a razão constante de 1:32 entre IP externo e interno.

    \begin{quadro}[htb]
        \begin{center}
            \caption{Mapeamento direto entre sub-redes internas/externas usando CGNAT 1:32.} 
            \label{tab:netmap}
            \vspace{0.2cm}
            \footnotesize
            \begin{tabular}{|c|c|c|}
            \hline
            Prefixo privado & Prefixo público & Quantidade de clientes \\
            \hline
            \hline
            ... & ... & ... \\
            /20 & /25 & 4096 \\
            /21 & /26 & 2048 \\
            /22 & /27 & 1024 \\
            /23 & /28 & 512 \\
            /24 & /29 & 256 \\
            /25 & /30 & 128 \\
            /26 & /31 & 64 \\
            /27 & /32 & 32 \\
            \hline 
            \end{tabular}
        \end{center}
        \centering{\small Fonte: do autor (2020)} 
    \end{quadro}

\subsection{Implementação de CGNAT no RouterOS}

    Até aqui, foi fundamentada a metodologia utilizada para o CGNAT. Nesta seção, será apresentada a implementação das regras em roteadores MikroTik através do sistema operacional embarcado RouterOS\footnote{RouterOS \url{https://mikrotik.com/software} é o sistema operacional nativo dos equipamentos MikroTik, sendo um software proprietário com \textit{kernel} Linux.}, com geração das regras através de ferramenta desenvolvida em Python de autoria própria.
    
    %\footnote{Python \url{https://www.python.org/} é uma linguagem de programação interpretada e multiplataforma.}.
    
    O primeiro passo para o desenvolvimento prático do CGNAT é entender o que o RouterOS nos proporciona para atingir esse objetivo. Por ser baseado em Linux, a ideia básica é construir as regras de NAT através do módulo de firewall do sistema, pois o NAT é interpretado como uma regra de firewall por fazer modificação no cabeçalho dos pacotes de dados ao alterar os valores de endereço e de porta originais.

    O módulo utilizado é a tabela NAT do firewall do RouterOS, sendo necessário configurar os seguintes parâmetros na geração das regras conforme definidos na documentação \cite{natmikrotik}:

    \begin{enumerate}[label=\alph*)]
        \item \label{nat:action} {\tt action}, especifica a ação que deve ser executada; neste caso, a ação definida foi o {\tt netmap}, que consiste no mapeamento direto entre as redes interna e externa;
        
        \item \label{nat:chain} {\tt chain}, é configurado como {\tt srcnat}, pois os pacotes alvos da regra originam-se na rede interna ao firewall; 
        
        \item \label{nat:protocol} {\tt protocol}, são configurados tanto TCP e UDP individualmente; 
        
        \item \label{nat:srcaddress} {\tt src-address}, é a sub-rede interna;
        
        \item \label{nat:toaddress} {\tt to-addresses}, é a sub-rede externa;
        
        \item \label{nat:toports} {\tt to-ports}, é a faixa de portas do IP público alocada.
    \end{enumerate}

    A única ressalva é que, para o pleno funcionamento do NAT para pacotes de protocolos de camada 3 (ICMP), devem ser criadas regras em que não estejam definidos os itens (\ref{nat:protocol} e (\ref{nat:toports}.

    Com os parâmetros definidos, o template do comando a ser executado na CLI é dado conforme o exemplo da Figura \ref{fig:netmap_cli}. O método de utilização simplificada do {\tt netmap} foi obtido de \cite{maia2018} e é uma técnica de CGNAT horizontal, ilustrada na Figura \ref{fig:cgnat_horizontal}. Neste exemplo, são utilizadas as sub-redes indicadas no exemplo do CON-PER-01 no Quadro \ref{tab:planilha_cgnat}. Pode-se notar que foi utilizado um prefixo /27 ao invés de um /22 no comando, isso acontece porque para que a regra de {\tt netmap} faça o mapeamento direto, as redes devem ter o mesmo prefixo para que a correspondência entre os IPs seja possível. Assim, uma rede /22 deve ser subdividida em 32 sub-redes /27 para que seja feito o mapeamento com uma rede pública de prefixo /27. Para fazer o casamento entre outros valores de prefixos, o princípio é o mesmo.

    \begin{figure}[!htb]
        \centering
        \caption{Template do comando de configuração de netmap no RouterOS.} 
        \label{fig:netmap_cli}
        
        \begin{Verbatim}[fontsize=\small]
            /ip firewall nat 
                add action=netmap           \
                chain=srcnat                \
                protocol=tcp                \
                src-address=100.64.0.0/27   \
                to-addresses=203.0.113.0/27 \
                to-ports=1536-3535          \
                disabled=yes
        \end{Verbatim} 

        {\small Fonte: do autor (2020).} 
    \end{figure}

    
    
    Acontece que, fazer geração manual de todas essas regras é um trabalho extenso e cansativo, que pode ficar sujeito a falhas humanas. Para isso, foi desenvolvido um utilitário de CLI para geração das regras para construir {\tt netmap} entre qualquer prefixo, desde que respeite a razão 1:32. O software foi desenvolvido em Pyhton.%, o que permite que seja usado em qualquer sistema operacional que tenha o ambiente de execução Python instalado.

    O fundamento utilizado na construção do programa baseia-se na geração de uma árvore binária oriunda das subdivisões recursivas de uma rede privada de determinado prefixo, situada na raiz, até que chegue em uma camada em que as folhas tenham o mesmo prefixo que a rede pública, procedimento ilustrado na Figura \ref{fig:arvore_binaria}. A partir daí, o algoritmo obtém uma lista com todas as sub-redes formadas pelas folhas dessa árvore, e gera as regras de {\tt netmap} conforme definido no template da Figura \ref{fig:netmap_cli}, incrementando o número de portas em um valor fixo de 2.000. 
    
    \begin{figure}[!htb]
        \centering
        \caption{Árvore binária das subdivisões recursivas de um /22 até um /25.}
        \label{fig:arvore_binaria}
        
        \small
        \begin{tikzpicture}[->,>=stealth',level/.style={sibling distance = 7cm/#1,
  level distance = 1.5cm}] 
            \node {/22}
            child {
                node {/23}
                child {
                    node {/24}
                    child {
                        node {/25}
                        node {/25}
                    }
                    child {
                        node {/25}
                        node {/25}
                    }
                }
                child {
                    node {/24}
                    child {
                        node {/25}
                        node {/25}
                    }
                    child {
                        node {/25}
                        node {/25}
                    }
                }
            }
            child {
                node {/23}
                child {
                    node {/24}
                    child {
                        node {/25}
                        node {/25}
                    }
                    child {
                        node {/25}
                        node {/25}
                    }
                }
                child {
                    node {/24}
                    child {
                        node {/25}
                        node {/25}
                    }
                    child {
                        node {/25}
                        node {/25}
                    }
                }
            };
        \end{tikzpicture}

        {\small Fonte: do autor (2020).} 
    \end{figure}
    
    A ferramenta desenvolvida foi nomeada como {\tt py-cgnat}. O código-fonte está disponível no GitHub\footnote{\url{https://github.com/williamabreu/py-cgnat}.} do autor e o executável pode ser baixado através do repositório oficial do Python\footnote{\url{https://pypi.org/project/pycgnat/}.}. O software foi desenvolvido seguindo as convenções do ecossistema Python e é \textit{open--source}, sendo lançado sob licença permissiva MIT para garantir liberdade de uso e de distribuição sem complicações. A versão 1.0b1 é a utilizada neste trabalho.
    
    
\subsection{Utilização do utilitário py-cgnat}
    
    A seguir, será utilizado o mapeamento do concentrador CON-PER-01 (conforme Quadro \ref{tab:planilha_cgnat}) para exemplificar o uso do programa {\tt py-cgnat} via terminal de comando. Para fazer a geração das regras de {\tt netmap} para RouterOS, o comando a ser executado no terminal segue conforme mostrado na Figura \ref{fig:pycgnat_gen}. Devem ser informados os seguintes parâmetros: a rede privada e a pública, a opção {\tt gen} para acionar o módulo de geração e a plataforma onde será configurado o CGNAT, que neste caso é o RouterOS. O último parâmetro é opcional e é o nome do arquivo de destino das regras caso queira ser salvo, pois se deixado em branco, as regras serão impressas no próprio terminal de comando. O campo plataforma foi posto por questão de extensibilidade, para deixar o software pronto para futuras novas versões, em que se pretende suportar equipamentos de outros fabricantes.
    
    \begin{figure}[!htb]
        \centering
        \caption{Exemplo de uso do programa py-cgnat para geração de CGNAT para RouterOS.} 
        \label{fig:pycgnat_gen}
        
        \begin{Verbatim}[fontsize=\small]
            In: pycgnat 100.64.0.0/22 203.0.113.0/27 gen routeros rules.rsc
            Out: null
        \end{Verbatim} 

        {\small Fonte: do autor (2020).} 
    \end{figure}
    
    O programa também calcula as traduções entre IP público-privado através do módulo {\tt trans}. Um exemplo é uma consulta para o IP privado 100.64.0.47, que retorna o IP público 203.0.113.15 e a faixa de portas definida entre 3536 e 5535 com o comando da Figura \ref{fig:pycgnat_transd}. De maneira reversa, uma consulta para o endereço 203.0.113.15 na porta 5000 retorna o cliente com IP 100.64.0.47, que pode ser verificada conforme Figura \ref{fig:pycgnat_transr}. As consultas de traduções independem de plataforma, pois fazem parte da lógica da metologia de mapeamento de portas utilizada. 
    
        \begin{figure}[!htb]
        \centering
        \caption{Exemplo de uso do programa py-cgnat para traduções de IP privado para público.} 
        \label{fig:pycgnat_transd}
        
        \begin{Verbatim}[fontsize=\small]
            In: pycgnat 100.64.0.0/22 203.0.113.0/27 trans -d 100.64.0.47
            Out: {"public_ip": "203.0.113.15", "port_range": [3536, 5535]}
        \end{Verbatim} 

        {\small Fonte: do autor (2020).} 
    \end{figure}
    
     \begin{figure}[!htb]
        \centering
        \caption{Exemplo de uso do programa py-cgnat para tradução de IP público para privado.} 
        \label{fig:pycgnat_transr}
        
        \begin{Verbatim}[fontsize=\small]
            In: pycgnat 100.64.0.0/22 203.0.113.0/27 trans -r 203.0.113.15:5000
            Out: {"private_ip": "100.64.0.47", "port_range": [3536, 5535]}
        \end{Verbatim} 

        {\small Fonte: do autor (2020).} 
    \end{figure}
    
    Além disso, o software foi desenvolvido para também ser utilizado como biblioteca de programação, com o objetivo de automatizar processos utilizando Python ou de integrar com sistemas já existentes, caso haja necessidade. Uma breve documentação (em inglês) do uso pode ser obtida no Apêndice A ou nos repositórios do {\tt py-cgnat} no GitHub ou no PyPI.
 
\section{Considerações sobre o uso de CGNAT}
    
    Uma ressalva para o processo de consulta das traduções de endereços é que, caso seja necessário descobrir qual cliente estava navegando com determinado IP em um determinado instante, é necessário consultar no banco do RADIUS qual o \textit{login} PPPoE esteve com aquele IP no dado momento. Esse tipo de solicitação acontece geralmente por ordem judicial ao ISP, a fim de descobrir a identidade de criminosos, que atuam na internet ou através dela.
    
    É muito importante seguir a documentação exemplificada na planilha do Quadro \ref{tab:planilha_cgnat} para garantir o cumprimento do Marco Civil da Internet. Por isso, a utilização de faixas de endereços distintas para clientes com dívida, denominado \textit{pool} de bloqueio, é considerada uma má prática quando não aplicada a técnica de CGNAT em clientes com velocidade de navegação reduzida, porque no período de bloqueio por dívida, seria impossível fazer rastreamento por IP. 

\subsection{Dificuldades encontradas na implantação}
    
    A dificuldade encontrada no \textit{deploy} do CGNAT foi que dois concentradores não estavam com a tabela NAT do firewall funcionando, sendo que o motivo da falha era desconhecido pela equipe. A solução foi subir as regras no dispositivo do ABR, pois como eram poucos clientes, as regras da rede do CGNAT não impactou o desempenho do roteador de borda, sendo feito acompanhamento de consumo de CPU durante momentos de pico do dia (por volta de 20h) para constatar isso. Outra fonte de constatação foi a ausência de reclamações por clientes daquela área onde foi aplicada essa solução de contorno. Também é possível colocar roteadores MikroTik dedicados ao CGNAT intermediando concentrador e borda, mas não foi adotada esta técnica como finalidade do trabalho.
    