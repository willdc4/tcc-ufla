\begin{center}
  \normalsize{\textbf{ANEXO B -- Um e-mail de alerta de segurança enviado pelo CERT.br}}
\end{center}

\pagestyle{empty} % remover numeração

\begin{verbatim}
From: "CERT.br" <cert@cert.br>
Subject: Alerta: [AS 262488] Mikrotik Possivelmente Comprometido - SOCKS 4145
Date: 3 June 2019 09:54:05 GMT-3
To: contato@minasnet.net
Cc: cert@cert.br
Reply-To: cert@cert.br

Caro responsável,

Os IPs no log ao final dessa mensagem possivelmente são de dispositivos
Mikrotik em sua rede que foram comprometidos e que estão sendo abusados
intensamente para o envio de spam.

Esse comprometimento habilita o serviço SOCKS na porta 4145/tcp
que pode ser abusado para diversas atividades, principalmente para o
envio de spam.

Essas atividades estão consumindo recursos de sua rede e provavelmente
incluindo seus IPs em listas de bloqueio.

Se você não for a pessoa correta para corrigir este problema, por
favor repasse essa mensagem para alguém da sua organização que possa
fazê-lo.

Gostaríamos de solicitar que:
1. cada dispositivo associado aos IPs abaixo fosse revisado e, se
  confirmada a suspeita de comprometimento, o problema seja resolvido
  (com a desativação do serviço SOCKS, alteração das senhas e atualização
   do RouterOS);
2. aumente-se o nível de monitoração da rede para determinar se outros
  dispositivos da sua rede também estão sofrendo do mesmo problema.

Sugestões de como realizar esses dois itens seguem abaixo.

* Como resolver o comprometimento dos dispositivos Mikrotik?
 1. Verifique a existência de um serviço SOCKS atendendo na
    porta 4145/tcp, executando o seguinte comando:
    /ip socks print
    Se o serviço estiver marcado como habilitado (enabled = yes),
    desabilite-o com o seguinte comando:
    /ip socks set enable=no
 2. Atualize a versão do Router OS para a última versão
    "Long-term/bugfix" ou "Stable/current", de acordo com as instruções
    do fabricante disponíveis na seguinte URL:
    https://wiki.mikrotik.com/wiki/Manual:Upgrading_RouterOS
 3. Apenas depois de atualizar o sistema altere a senha com o comando
    abaixo:
    /user set USUARIO password=NOVA_SENHA
    onde USUARIO é o usuário utilizado para conectar no mikrotik

* Como identificar outros dispositivos sendo abusados pela mesma
 técnica?
 Sugerimos também que monitore regularmente o tráfego de sua rede
 através do uso de netflow para identificar esse problema. Uma
 sugestão seria monitorar o tráfego na porta 4145/tcp e observar o
 aumento anormal de conexões com destino às portas 25/tcp e 587/tcp.

* O que é o CERT.br?
 O CERT.br -- Centro de Estudos, Resposta e Tratamento de Incidentes
 de Segurança no Brasil -- é o Grupo de Resposta a Incidentes de
 Segurança para a Internet brasileira, mantido pelo NIC.br do Comitê
 Gestor da Internet no Brasil.  É o grupo responsável por tratar
 incidentes de segurança em computadores, envolvendo redes conectadas
 à Internet brasileira.

 IP              | ASN    | Porta | Status | Timestamp            | Resultado do 
                                                                    Teste
=============================================================================
177.66.52.134    | 262488 |  4145 | OPEN   | 2019-06-03T08:50:01Z | SMTP banner: 
                                                                    confirmed
177.66.52.137    | 262488 |  4145 | OPEN   | 2019-05-27T18:02:30Z | SMTP banner: 
                                                                    confirmed
177.66.52.195    | 262488 |  4145 | OPEN   | 2019-05-28T20:55:06Z | SMTP banner: 
                                                                    confirmed
177.66.52.232    | 262488 |  4145 | OPEN   | 2019-05-28T20:55:17Z | SMTP banner: 
                                                                    confirmed
177.66.52.252    | 262488 |  4145 | OPEN   | 2019-05-29T22:44:48Z | SMTP banner: 
                                                                    confirmed
179.106.162.108  | 262488 |  4145 | OPEN   | 2019-05-27T23:52:08Z | SMTP banner: 
                                                                    confirmed
179.106.162.145  | 262488 |  4145 | OPEN   | 2019-05-27T18:04:40Z | SMTP banner: 
                                                                    confirmed
179.106.166.57   | 262488 |  4145 | OPEN   | 2019-05-30T14:10:38Z | SMTP banner: 
                                                                    confirmed
179.106.166.92   | 262488 |  4145 | OPEN   | 2019-05-27T18:04:41Z | SMTP banner: 
                                                                    confirmed
=============================================================================

Cordialmente,
--
CERT.br/NIC.br
<cert@cert.br>
https://www.cert.br/
\end{verbatim}