\chapter{ATIVIDADES REALIZADAS}

    Descrever a plenitude das atividades executadas no estágio seria extremamente extenso, pois o estágio na área de TI demanda a execução de inúmeras tarefas corriqueiras. Por isso, aqui serão descritas as principais atividades desenvolvidas no período do estágio em administração de serviços de redes, sendo elas a fundamentação das técnicas adquiridas através de capacitações e de treinamentos na empresa, o processo de endereçamento IP e implementação de CGNAT para provisionamento de acesso à internet, o deploy de um serviço de VPN e o desenvolvimento de regras de bloqueio por firewall. 

\section{Capacitações e treinamentos}

    Conforme definição de dicionário, treinar consiste em praticar regularmente qualquer atividade e capacitar é tornar-se apto à atividade \cite{michaelis2015}. Ou seja, primeiro vem a capacitação e por conseguinte o treinamento. Portanto, a capacitação é a primeira tarefa do estagiário dentro da organização, para nivelamento de conhecimento, e o treinamento é uma constante durante todo período de estágio supervisionado, pois o estudante estará aplicando, aprendendo e desenvolvendo habilidades práticas no ambiente da organização em todo o decorrer, sob supervisão e suporte de um profissional. A capacitação e o treinamento oferecido ao estagiário é um dos principais objetivos da atividade de estágio em si.
    
    Na Minasnet, após a admissão de qualquer colaborador, seja tanto trabalhador formal quanto estagiário, é aplicada capacitações e feito treinamentos para integrar o novato nos processos da empresa, de acordo com a função que este for assumir. Para o estagiário, nas primeiras semanas de trabalho um dos integrantes do NOC fornece uma capacitação individual expositiva, apresentando conceitos básicos de redes e da topologia do backbone da empresa, como também de procedimentos de atendimento e de suporte a clientes finais, para entender quem são os clientes da empresa. São fornecidos manuais dos equipamentos que são usados e o novato é encaminhado para um treinamento para praticar os procedimentos aprendidos. O primeiro treinamento foi montar um pequeno provedor de laboratório utilizando equipamentos Mikrotik, simulando o roteamento estático e dinâmico com OSPF em algumas RB750 e enlaces de rádio com Groove, tudo realizado no primeiro dia do estágio, para se familiarizar desde então com o RouterOS.
    
    Mais adiante, foi solicitado a matrícula e exigida a apresentação de conclusão do minicurso ofertado pela Fundação Bradesco na modalidade EaD denominado ``Fundamentos de ITIL'', no qual apresentou os conceitos básicos sobre o framework ITIL para planejamento estratégico em infraestrutura de TI, com carga horária total de 16h. O último treinamento oferecido foi o minicurso presencial e com duração de 18h denominado ``Protocolo de Roteamento OSPF e Mikrotik de Iniciante a Intermediário'', restrito aos técnicos internos da Minasnet, para treinamento de roteamento dinâmico com OSPF no Mikrotik, utilizando o simulador GNS3 bem como dispositivos reais fornecidos.
    
    Demais conhecimentos adquiridos foram obtidos com capacitação individual, em que um colega da equipe ensinava determinado procedimento e depois acompanhava a atuação, como também através de cursos e de materiais digitais compartilhados pela equipe ou pela própria iniciativa de buscar conteúdo em fóruns e em plataformas vídeo.
    
